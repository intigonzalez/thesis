\thispagestyle{plain}
\addcontentsline{toc}{chapter}{To the reader about contributions}
\noindent\makebox[\linewidth]{\rule{\textwidth}{2pt}}

\vspace{0.5cm}
{\Huge \textbf{To the Reader}}
\vspace{0.2cm}

\noindent\makebox[\linewidth]{\rule{\textwidth}{2pt}}

\newcommand{\placelast}{west}
\newcommand{\placelastt}{south}
\newcommand{\placelasttt}{north}

In the first two chapters, we discuss the relevance of resource-aware programming in the development of software systems.
It is also highlighted that resource awareness requires extensive support from the runtime environment.
Severe limitations in existing approaches for resource accounting and reservation prevent the use of resource-aware programming.
In particular, we show how the granularity level at which a resource accounting method can be applied is of utmost importance. % because if affects applications' behavior.
This is further discussed in Chapter~\ref{chap:abstractions_and_resource_management}, where we highlight the fact that every time a new software abstraction is created, it can be seen as a new granularity level.
Since creating abstraction is common in software development, mechanisms to easily develop support for resource-aware programming are needed. 

In the rest of this thesis, we present three approaches that contribute to achieve our goal -- supporting resource-aware software development.
Figure~\ref{fig:subway-map} depicts a subway map to establish how the new mechanisms we propose are connected to each other, how they contribute to the common goal, and their relationship with other approaches.
Although this metaphor is by no means complete, it helps to quickly summarize what is required to achieve resource awareness. 

{
\fontfamily{phv}\selectfont

\begin{figure}[!ht]
\centering
\begin{tikzpicture}

% main railway
\draw[draw=red!70, line width=9pt, cap=round, join=round] (0,0)--(3,-3)--(7.5,-3)--(8.5,-4)--(9.5,-5)--(11,-5);
% abstractions' railway
\draw[draw=blue!40, line width=9pt, cap=round, join=round] (7.5,-4.5)--(5.5,-4.5)--(5.5,-2.65)--(7.5,-2.65)--(9,-1.65)--(10.7,-1.65);
% monitoring railway
\draw[draw=green!40, line width=9pt, cap=round, join=round] (3.95,0.45) -- (2.45,0.45)--(1.25,-0.75)--(2.25,-1.75) -- (3.25,-0.75)--(4.75,-0.75);
% components railway
\draw[draw=pink, line width=9pt, cap=round, join=round] (2,-4.35)--(1,-5.35)--(3,-3.35)--(4.2,-3.35)--(4.2,-4.35)--(4.2,-5.65);

% main railway
\foreach [count=\c] \x/\cy/\text/\place in {
   0/0/{}/east,
   8.5/-4/\parbox{2.5cm}{Ease construction of tools}/west,
   11/-5/\parbox{1.7cm}{Resource awareness}/\placelast
  } {
  
  \filldraw[fill=white, draw=black, line width=1pt] (\x,\cy) circle [radius=0.15];
  \draw[draw=black,dotted, line width = 1pt] (\x,\cy) node[anchor=\place, font=\scriptsize, inner sep=0.3cm] {\text};
}

% abstractions' railway
\foreach [count=\c] \x/\cy/\text/\place in {
   7.5/-4.5/\parbox{1cm}{{Internal DSL}}/north west,
   6.5/-4.5/\parbox{1cm}{{External DSL}}/north,
   5.5/-4.5/\parbox{0.7cm}{{API}}/north,
   7.5/-2.65/\parbox{0.5cm}{{Fluent API}}/south east,
   9/-1.65/\parbox{1.7cm}{Metamodeling}/south,
   10.7/-1.65/\parbox{1.9cm}{{Components}}/\placelast
  } {
  
  \filldraw[fill=white, draw=black, line width=1pt] (\x,\cy) circle [radius=0.15];
  \draw[draw=black,dotted, line width = 1pt] (\x,\cy) node[anchor=\place, font=\scriptsize, inner sep=0.3cm] {\text};
}

% monitoring railway
\foreach [count=\c] \x/\cy/\text/\place in {
   3.95/0.45/{MRTE Modification}/west,
   2.45/0.45/{Instrumentation}/south,
   3.25/-0.75/\parbox{1.6cm}{Adaptive Monitoring}/north west,
   4.75/-0.75/{Sampling}/\placelast
  } {
  
  \filldraw[fill=white, draw=black, line width=1pt] (\x,\cy) circle [radius=0.15];
  \draw[draw=black,dotted, line width = 1pt] (\x,\cy) node[anchor=\place, font=\scriptsize, inner sep=0.3cm] {\text};
}

% component railway
\foreach [count=\c] \x/\cy/\text/\place in {
   2/-4.35/{Contracts}/east,
   1/-5.35/{Kevoree}/east,
   4.2/-4.35/{cgroups}/north east,
   4.2/-5.65/\parbox{2cm}{System\\ Reconfiguration}/\placelasttt
  } {
  
  \filldraw[fill=white, draw=black, line width=1pt] (\x,\cy) circle [radius=0.15];
  \draw[draw=black,dotted, line width = 1pt] (\x,\cy) node[anchor=\place, font=\scriptsize, inner sep=0.3cm] {\text};
}

%joins
\foreach [count=\c] \x/\cy/\text/\place in {
   1.12/-0.88/\parbox{1.6cm}{Resource Accounting}/east,
   2.12/-1.88/\parbox{1.8cm}{Lightweight accounting}/east,
   3/-3.15/\parbox{2cm}{Accounting\\ for components}/east,
   4.2/-3.15/\parbox{2cm}{Reservation for components}/north,
   5.5/-2.85/\parbox{2cm}{Working with\\ Abstractions}/\placelastt
  } {
 
  \filldraw[fill=white, draw=black, line width=0.04cm] (\x,\cy) circle [radius=0.3];
  \filldraw[fill=white, draw=black, line width=0.04cm] (\x,\cy) circle [radius=0.13];

  \draw[draw=black,dotted, line width = 1pt] (\x,\cy) node[anchor=\place, font=\scriptsize, inner sep=0.3cm] {\text};
}

% contributions
\path (8.5,-4) node[anchor=south east, inner sep=-0.04cm] {
\includegraphics[width=20pt, height=20pt]{library}
}
(3,-3.15) node[anchor=south east, inner sep=-0.01cm] {
\includegraphics[width=20pt, height=20pt]{library}
}
(4.2,-3.15) node[anchor=south east, inner sep=-0.01cm] {
\includegraphics[width=20pt, height=20pt]{library}
}
(11,-5) node[anchor=south east, inner sep=-0.06cm] {
\includegraphics[width=20pt, height=20pt]{goal}
}
;
% indicator of current location
\draw[color=gray,draw=yellow, line width=7 pt] (0,0) circle (0.4cm);
\path
    [
		postaction={
	        decorate,
            decoration={
                raise=-9pt,
                text along path,
                text align/fit to path stretching spaces=true,
                reverse path=true,
                text align/align=center,
                text align/left indent={1cm}, % \pi * radius
                text align/right indent={0.0cm},
                text={|\scriptsize|YOU ARE HERE}
            }
        }
    ]
(0,0) circle (0.6cm);

% legend
\path (11, 1.9) node {
	\includegraphics[width=20pt, height=20pt]{goal}
} (12.2,1.8) node {{\scriptsize Objective}}
;
\path (11, 1) node {
	\includegraphics[width=20pt, height=20pt]{library}
} (12.4,0.9) node {{\scriptsize Contributions}}
;
\draw[line width=0.1cm, draw=red!70] (10.7, 0.2)--(11.3,0.2) (12.4,0.2) node {{\scriptsize Thesis's way}}
;


\end{tikzpicture}
\caption{This subway map shows how our research contributes to the state-of-the-art on supporting resource-awareness.} \label{fig:subway-map}
\end{figure}

}


This thesis makes three \textit{stops}.
%\thispagestyle{empty}
\begin{description}
\item[Resource accounting for components] The \textbf{first} \textit{stop} is to propose an approach for resource consumption monitoring in component-based systems that run on top of MRTEs (see  Chapter~\ref{chp:scapegoat}).
Before it, our journey takes us through a way where we collect the basic methods needed to perform resource accounting.
Afterwards, at the first \textit{stop}, we reuse such methods to provide \textit{efficient} accounting for components.

\item[Resource reservation for components] Resource accounting is the foundation for a safe \textbf{second} \textit{stop} -- providing resource reservation for components.
It is then when we discuss a methodology to delay the selection of the binding method used to guarantee resource reservation, until the deployment phase at runtime.
This serves both to guarantee low performance overhead and resource reservation per component (see  Chapter~\ref{chp:squirrel}).

\item[Ease the constructions of tools] The journey continues, and we learn how the mechanisms to define software abstractions lack support for building resource accounting tools.
We also realize that this issue makes us remember our first \textit{stop} because components are just a concrete abstraction; and it also brings memories of the second one because new methods to calculate resource consumption may influence the mechanism we choose to represent components at runtime. 
Then we \textit{stop} for the \textbf{third} time; we propose an approach to ease the construction of efficient and customized memory profilers for MRTEs (see  Chapter~\ref{chp:dsl-memory}).
\end{description}

\newpage\thispagestyle{empty}\addtocounter{page}{-1}



