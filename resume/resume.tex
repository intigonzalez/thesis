\pagestyle{empty}
\vspace{-2cm}
%---------------------------%
\section*{Résumé}
Aujourd'hui, les syst\`emes logiciels sont omnipr\'esents.
Parfois, les applications doivent fonctionner sur des dispositifs à ressources limit\'ees.
Toutefois, les applications nécessitent un support d'exécution de faire face à de telles limitations. 
Cette thèse aborde le problème de la programmation pour créer des systèmes «conscient des ressources» supporté par des environnements d'exécution adaptés (MRTEs).
En particulier, cette thèse vise à offrir un soutien efficace pour recueillir des données sur la consommation de ressources de calcul (par exemple, CPU, mémoire), ainsi que des mécanismes efficaces pour réserver des ressources pour des applications spécifiques.
Dans les solutions existantes, nous trouvons deux inconvénients importants.
Les solutions imposent un impact important sur les performances à l'exécution des applications.
La création d'outils permettant de gérer finement les ressources pour ces abstractions est encore une tâche complexe.
Les résultats de cette thèse forment trois contributions:
\begin{itemize}
\item Un cadre de surveillance des ressources optimiste qui réduit le coût de la collecte des données de consommation de ressources.

\item Une méthodologie pour sélectionner les le support d’exécution des composants au moment du déploiement afin d'effectuer la réservation de ressources.

\item Un langage pour construire des profileurs de mémoire personnalisées qui peuvent être utilisés à la fois au cours du développement des applications, ainsi que dans un environnement de production.
\end{itemize}

\section*{Abstract}

Software systems are more pervasive than ever nowadays.
Occasionally, applications run on top of resource-constrained devices where efficient resource management is required; hence, they must be capable of coping with such limitations.
However, applications require support from the runtime environment to properly deal with resource limitations.
This thesis addresses the problem of supporting resource-aware programming in execution environments.
In particular, it aims at offering efficient support for collecting data about the consumption of computational resources (e.g., CPU, memory), as well as efficient mechanisms to reserve resources for specific applications.
In existing solutions we find two important drawbacks.
First, they impose performance overhead on the execution of applications.
Second, creating resource management tools for these abstractions is still a daunting task.
The outcomes of this thesis are three contributions:
\begin{itemize}
\item An optimistic resource monitoring framework that reduces
the cost of collecting resource consumption data.

\item A methodology to select components' bindings at deployment time in order to perform resource reservation.

\item A language to build customized memory profilers that can be used both during applications' development, and also in a production
environment. 
\end{itemize}
