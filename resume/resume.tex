\markboth{}{}
\pagestyle{empty}
\vspace{-2cm}
%---------------------------%
\section*{Résumé}
Aujourd'hui, les éditeurs logiciels ne conçoivent, développent et ne maintiennent plus leur offre logicielle avec comme cible un client unique. Au contraire, les offres logicielles sont conçues pour cibler plusieurs entités. Par conséquent, ces applications doivent s'intégrer dans des environnements différents et s'adapter aux besoins des clients. Ainsi, les produits logiciels développés ne sont plus des programmes uniques, mais des familles de produits \cite{parnas1976design}.
Les systèmes configurables facilitent la création de ces familles de produits. Grâce à eux il est possible de créer un produit logiciel en sélectionnant les fonctionnalités qui seront intégrées. Cependant, la validation de ces systèmes  est une tâche complexe. Un système configurable   peut  générer plusieurs millions de configurations possibles. Il ne s'agit donc plus de valider un seul et unique produit, mais un ensemble de produits. Cet important nombre de configurations est un problème pour les personnes  chargées de la validation.
Nous proposons trois contributions qui visent à mieux répondre aux problématiques liées à la variabilité lors des projets de test. 
\begin{itemize}
\item  une présentation détaillée de deux projets de test industriels faisant face à des problématiques de variabilité issus de deux entreprises : Cisco et Orange.

\item une méthode originale basée sur les techniques de programmation par contraintes pour extraire des configurations de test qui respectent le critère Pairwise à partir d'un modèle explicite de la variabilité. 

\item une comparaison de cette approche par rapport aux techniques de l'état de l'art et une étude de l'application de cette technique de test sur deux projets de tests industriels.
\end{itemize}


\section*{Abstract}

Nowadays, software companies develop and maintain their software for several clients. Consequently, these applications have to  be integrated in heterogenous context and adapt to the user requriements. All these products are sharing commonalities but also differ in certain point due to business specific constraints.
Configurable systems facilitate the creation of these product families. With them it is possible to create a software product by selecting the features that will be integrated, thus, the creation of a product is greatly simplified. However,  the validation of these systems is a complex task. A configurable system can generate millions of possible configurations. Thus, validation process doesn't consist in validating a single product but in validating a set of products. This large number of configurations is a problem for those responsible of the validation.
In this thesis we propose three contributions that aim to solve issues raised by  variability during test projects : 

\begin{itemize}


\item  A detailled presentation of two industrial test projects coping tat variaibility issues 


\item An original methodology based on constraint programming techniques to select test configurations that respect pairwise criteria from a feature model


\item An exhaustive comparison of this approach with the existing approches and a detailled study of the application of a such techniques on the two industrials projects.
\end{itemize}
