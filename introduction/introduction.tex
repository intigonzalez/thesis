%------------------------------%
%------------------------------%


\chapter{Introduction}
%\addcontentsline{toc}{chapter}{Introduction}
%------------------------------%
%------------------------------%
\markboth{Introduction}{Introduction}

\coolphrase{
This is the text
}{inti}

\section{Motivations and Overview}

Software systems are becoming even more pervasive nowadays.
From simple applications for controlling home appliances to complex tools for dealing with productive processes, software systems are now found in any environment.
The end-user's expectations have also grown along the development of the software industry.
For instance, installing several applications provided for non-fully trusted sources in a mobile phone or closing a window in your office through a website are considered nowadays routinary tasks which only require engineering effort.
However, the software/hardware industry faces several problems while coping with these expectations.
In the context of this thesis, three of them are worth mentioning: the need to handle millions of users, the ubiquity of resource-constrained devices, and the
open nature of execution environments.   

The widespread usage of interconnected devices and the trend to use client/server architectures have led to a massive increment in the number of users some applications have to deal with.
To support so many users, it is necessary to manage a massive computing infrastructure which is both complex and extremely costly.
The emergence of cloud services was a \textit{natural} development of these infrastructures.
Cloud services is all about providing access to computing resources and it is common to use them to build client/server applications.
Cloud services are shipped in many forms (e.g., virtual machines, software) but in general we can reduce them, using a rather simplistic model, to basic resources such as CPU cycles, memory, network bandwidth and storage capacity.
This simplistic view is useful if we consider how important is for cloud services providers to reduce the usage of resources in order to increase efficiency.
For instance, given the large nature of the cloud, a provider of platform as a service (PaaS) can reduce the operational costs if it manages to squeeze even a few thousand of bytes from each request.
As a consequence, carefully managing resources such as CPU and memory becomes a primary concern for cloud service providers.

Despite of the continuous evolution of hardware, applications are always capable of demanding more computational resources.
This results in a limitation in computing resources that is evident in wearable devices, but in the near future we can expect to see the same kind of concern in the emergent \textit{Internet of Things}.
In this kind of devices it is fundamental to offer tools for sharing the scarce resources among all the applications.
It is worth noting that managing resources has been a central problem in computer science and software engineer for a long time.
However, I highlight the problem in this context for two reasons.
First, because I think that new technologies to build applications for these devices allow better resource management techniques.
Second, because it shows that limitations in resource quantity is still a concern for software developers.

Open execution environments allow the deployment of new applications provided for different sources at any time, being these sources trustworthy or not.
Since end-users are able to install many applications that share the execution environment, it is impossible for the developers of the runtime or for the developers of a single application to predict under what condition an application will be executing.
Although not new, this problem is specially important because now it is possible to use these execution environments to build actuators that operate on the physical environment.
A possible source of failure in these applications is the lack of resource isolation.
In short, an application makes another to fail by consuming the resources needed for the later.

The common concern in the three previous problems is that of making applications and execution environments aware of and capable to cope with resource limitations. 
This thesis addresses the problem of supporting resource-aware programming in execution environments.
Given their popularity, it focuses on facing the issue of resource-awareness in managed runtime environments (MREs) such as Java and the CLR.

Existent solutions for resource management in MREs have two fundamental problems.
First, they impose a significant performance overhead while monitoring the resource consumption of applications.
Second, despite of the broad usage of MREs to execute component-based applications and to build domain-specific 
abstractions, it is still had to provide tools to seamless deal with the resource management concern.

\if 0 


\subsection{General Objective}
Offering portable support for \textbf{resource-aware programming} in \textbf{managed runtime environments}.

Disminuir el costo de gestionar recursos en tiempo de ejecucion y facilitar el uso de tecnicas de resource-aware programming en ambientes de ejecucion manajeados.

Ofrecer soporte para la resource-aware programacion en ambientes de ejecucion manejados. 
\todo{Soporte es un termino ambiguo, lo de portable tambien es dificil de enteder en el contexto}

\subsection{Specific Objectives}
\begin{enumerate}
\item Reduce the amount of CPU and memory used to determine the resources consumed by different parts of an application built to execute on top of managed runtime environments.\\
Reducir la cantidad de CPU y memoria utilizada para determinar los recursos consumidos por diferentes partes de una applicacion construida para ejecutarse 
sobre un ambiente de ejecucion manejado.\\

\item Provide abstractions for supporting the development of tools to efficiently measure at runtime the memory consumed by domain-specific abstractions that were implemented atop the \textit{classic} object-oriented paradigm.

Proveer abstracciones que permitan crear
herramientas eficientes para medir en tiempo de ejecucion la memoria consumida por abstracciones de dominio especifico que fueron implementadas usando el clasico paradigma orientado a objetos.\\

\todo{Preguntar las 4 preguntas y si el como va en el objetivo especifico. Mami como yo conformo un objetivo.. el problema que yo veo aqui es que no es facil medir lo e aligerar el desarrollo. Necesito estudio con usuarios  eso no lo hice.}
\item Reduce the complexity and computational cost of managing resource reservation in component-based software systems.

Reducir la complejidad y costo computacional de gestionar la reserva de recursos en sistemas basados en componentes.
\end{enumerate}

\subsection{Research questions}

\subsection{Research justification}

\subsection{Viabilidad de la investigacion}

\subsection{Contributions}
\begin{enumerate}
\item A methodology for resource-management-aware component deployment.
\item An optimistic resource monitoring approach that reduces the performance overhead.
\item A domain-specific language to developed custom memory profilers.
\end{enumerate}

En la introduccion, ademas de declarar cuales son los objetivos de la tesis y las contribuciones de la misma, debo hacer una breve recapitulacion de los elementos fundamentales a tratar. La estructura es por supuesto la tradiccional de contexto, problema, solucion propuesta y como la evaluamos. Sin embargo, es quizas aportuno agregar algo como impacto esperado de esta investigacion. 

\fi

\section{Dissertation Structure}