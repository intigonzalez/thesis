\section{Perspectives}

% arreglar y creerme esta oracion
The work presented in this thesis represents a step towards proving support for resource aware programming.
This work presents many perspectives which are presented below. 


%\subsection{Scapegoat}

\paragraph{Reducing overhead of instruction accounting}

% este puede ser, pero esta escrito de forma horrible
Instrumentation by bytecode rewriting induces a particularly high performance overhead, specially when used for instruction accounting.
Despite the use of adaptive monitoring which reduces the performance overhead, there are occasions when the overhead imposed is still high: while doing localized monitoring, and while probes are being activated.
A way to reduce both overheads is by identifying sections of code that do not need to be instrumented.

We can learn at runtime how many instructions a method executes for a given input.
This way, we only need to instrument some methods a few times until we find a predictive model (as in machine learning) that is able to predict the number of executed instructions~\cite{tesauro2006hybrid}.
Afterwards, no instrumentation code is added to such methods and their consumption is measured by evaluating a prediction model when they are called.
 
%In addition, we plan to study alternatives to improve instruction accounting. %These alternatives are about using peak monitoring and learning monitoring.
%For example, we plan to study the use of machine learning for monitoring \cite{tesauro2006hybrid}. Based on a machine learning approach, it is possible to train the monitoring system to do the instruction instrumentation. Then, instead of doing normal instruction instrumentation, we might only do, for example, method-calls instrumentation and with the learning data, the monitoring system should be able to infer the CPU usage of each call, whilst lowering the overhead.

\paragraph{Response to misbehavior}
% no me gusta esto
%Scapegoat currently uses code injection at load-time to perform fine-grained monitoring. 
%The adaptive monitoring approach we have presented provides good results, but we believe we can reduce the overhead of CPU and memory monitoring by using a modified JVM and injecting specialized bytecode to cooperate with it.
%The modified JVM would account for the resources at a low-level, while the instrumentation code could provide application-level information like the component boundaries. 
%This should result in a more efficient solution than calculating resource usage at the application-level only.

% puede ser, hablar un poco tambien de la deteccion de problemas en el uso del canal

%Reacting to events about resource consumption is of utmost importance to guarantee the robustness of a system.
Resource accounting only provides a single step to support the reconfiguration of a system when events about resource consumption are triggered.
Handling such events in the proper way, eliminating the source of misbehavior and guaranteeing  consistency  of the system, is of utmost importance.
There are approaches to face similar issues, for instance, replacing a service for an alternative implementation when the response time of the former is high.
Since many responses are possible, it is worth considering a systematic approach to select them using a heuristic instead of a hard coded policy.
This heuristic may choose among a set of reconfiguration policies that include limiting the resources available to a component, replacement of components, slowing down a component by simply delaying the access to its interface, and moving components across the distributed infrastructure

%A second research perspective consists in proposing appropriate reactions when the source of a problem is discovered by Scapegoat. 
%Indeed, reconfiguration policies when a resource-consumption problem is found could range from resource limitations for faulty components, to a replacement of the component or of part of the application, to degrading the applications functionality.
%In the context of distributed systems, the set of possible reconfigurations is larger and can include moving components across the distributed infrastructure.
%It is necessary to choose how to efficiently reconfigure the system to deal with the discovered fault.

\paragraph{Applying the Squirrel methodology to domains with strong safety and security concerns}

% estpo primero es interesante
%In the future, we envision augmenting the Squirrel framework with automatic and dynamic reasoning capabilities in order to automatically place and migrate components with resource contracts over a distributed architecture.
%This paper also opens interesting perspectives by applying the presented concepts in different application domains with strong security and safety concerns.

Choosing the mapping from high-level concepts to low-level system abstractions is one of the step in providing a concrete implementation of a component model.
As this mapping may affect the performance overhead of a system and its capacity to provide resource reservation, it can also affect other properties, such as security.
In~\cite{Gama:2010:SCS:2176905.2176915}, the authors proposed an approach to execute components in a sandbox when it is not possible to trust in their origin; using these sandboxes also has an impact on the overall performance of the system.
As a consequence, we can consider that the appropriate mechanism to put a component in a sandbox should be selected at deployment time.



%esto esta visto, pero se puede mencionar
%Applying the concepts in this paper to other application domains with strong safety and security concerns is of interest.
%For instance, security concerns often leverage resource isolation as a means to avoid sharing confidential information with untrusted components within the same container.
%\hl{WHAT?: Indeed, security concerns are often correlated with resource isolation in managing that some confidential information cannot be shared to untrusted components within the same container. }

% esto no se lo que es
%The possibility of applying a pattern-based approach to increase an application's security without compromising its efficiency looks promising~\cite{DBLP:conf/kbse/MorinMFTBJ10}.



%Energy efficient dynamic adaptation, to switch off parts of the electronics depending on the resources usage

%\subsection{Custom memory profilers}
%In the future, we plan to address the limitations of the execution model we propose.
%In particular, we are aware that it is not as powerful as other query languages since it is based on a single traversal of the  graph.
%The advantages of the chosen approach are that it guarantees a low impact on the performance and it is easy to weave with existent technique to explore the heap.
%An alternative we consider is to leverage the results from graph databases.
%It would provide two benefits: i) algebraic transformation of the queries with all the potential optimizations and, ii) an already known language for developers.


\paragraph{A language to manipulate the graph of objects}
% 2 - aumentar el lenguage para que pueda manipular el grafo de objetos
Exploring the graph of objects may be useful to reveal bugs in a system.
In particular, to detect memory leaks and excessive memory consumption.
It has also been discussed how to evaluate assertions on some data structures in memory by simply traversing the objects in the heap~\cite{Reichenbach:2010:GCE:1869459.1869482}.
The idea is that exploring the heap may be considered a cross-cutting concern.
In this context, it is interesting to study whether modifying the graph of objects may help to solve problems that can be identified by looking at objects and how they are connected.

One interesting example is eliminating memory leaks; in \cite{dsn:15:attouchi:incinerator}, the authors proposed a mechanism to remove stale references in dynamic OSGi applications, the approach is largely based on eliminating references between objects once the JVM detects the stale references.
Since this is simply a modification to the object graph, it is worth considering the use of a language to express this kind of \textit{automatic}
bug fixing without having to hard code them using a low-level language nor modifying the JVM.
The question is to what extent the same goals can be achieved without modifying the JVM by simply using profilers API such as JVMTI.

% 3 - cambiar su semantica para que en lugar de un solo pase sea mas como un dataflow analysis

\paragraph{Generate the specification of memory profilers from models that describe a domain-specific abstraction}
% 4 - crear un mecanismo para que la definion del profiler a partir de la especificacion de la abstraccion se haga de una forma mas automatica, seria genial
Automating the construction of memory profilers for new software abstractions may ease software maintenance.
Using our approach, engineers have to implement both the software abstraction and the memory profilers.
A way to further reduce the development effort is by using high-level descriptions of the abstraction to generate the definition of a profiler in our language.
This way, engineers could focus on the definition of the abstraction. 

For instance, this is the approach followed by Xtext~\cite{Eysholdt:2010:XIY:1869542.1869625} to automatically generate debuggers for languages that inherit from the base language (Java).
The idea is using models, as in \gls{MDSD}~\cite{Stahl:2006:MSD:1196766, Fowler:2010:DSL:1809745}, to describe the software abstraction and also the way a concrete instance is mapped to a low-level technology.
In particular, this can be done using the metamodel of the abstraction (as in a DSL) and a traceability model to see how a concrete model is transformed to, for instance, Java.

\paragraph{Use declarative languages to define profilers}
% 1 - tener un lenguage mas declarativo, tipo CYPHER
Declarative languages are often preferred for solving tasks such as querying a data structure.
In the case of querying a graph, there are languages able to express complex requests in concise ways, for instance, CYPHER.
As we already discuss, the problem is how to efficiently schedule the execution of queries written in such languages.

One interesting path to explore is reusing a subset of these languages that can be efficiently implemented despite the MRTEs' constraints.
The main challenges in implementing an alternative like this will be to create a scheduler to optimize th execution of such queries.
This is important because the graph of objects is not explicitly represented in a MRTE.


% 5 - lenguage similar para otros tipos de recursos