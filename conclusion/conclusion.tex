%------------------------------%
%------------------------------%
\chapter{Conclusion et Perspectives}
\markboth{Conclusion et Perspectives}{Conclusion et Perspectives}
%------------------------------%
%------------------------------%
\label{chap:ccl}

\section{Conclusions}

\begin{comment}

\subsection{Transfert du monde académique vers l'industrie }
Les travaux présentés dans cette thèse se sont fait dans le cadre d'un financement CIFRE (Convention Industrielle de Formation par la Recherche). Une partie transfert vers l'industrie reste à effectuer. Nous envisageons d'industrialiser les développements tout en conservant une interaction avec le domaine de la recherche. Cette industrialisation passera par le développement d'un produit logiciel utilisable pour modéliser la variabilité et sélectionner les configuration de tests. Puis de déployer les solutions proposées sur des projets pilotes comme le test d'applications mobiles ou les applications testées doivent s’exécuter sur un large panel de téléphone mobile, puis chez des clients.

Les nombreux travaux de recherche en cours autour de la variabilité, comme le \textit{reverse engineering} de modèle de \textit{features}, répondent à des problématiques concrètes des industriels. La poursuite de ce partenariat académie-industrie serait bénéfique pour les deux parties. D'un coté, la fourniture de cas concret industriels aux chercheurs permet d'alimenter les thématiques de recherches et contribue d'autre part à la présentation de solutions innovantes pour l'entreprise.
\end{comment}