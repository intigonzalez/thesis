\selectlanguage{french}
%\markboth{}{}
%\pagestyle{empty}
\vspace{-2cm}
%---------------------------%
\chapter*{Résumé}
\markboth{Résumé en français}{Résumé en français}
\addcontentsline{toc}{chapter}{Résumé en français}

Aujourd'hui, les systèmes logiciels sont omniprésents. Ils se trouvent dans des environnements allant des contrôleurs pour appareils ménagers à des outils complexes pour traiter les processus industriels. Les attentes de l'utilisateur final ont grandi avec le développement de l'industrie, du matériel et des logiciels. Cependant, l'industrie doit faire face à plusieurs défis pour contenter ces attentes. Parmi eux, nous trouvons des problèmes liés à la question générale de traiter efficacement les ressources informatiques pour satisfaire aux exigences non fonctionnelles. En effet, parfois, les applications doivent fonctionner sur des dispositifs à ressources limitées ou des environnements d'exécution ouverts où la gestion efficace des ressources est d'une importance primordiale pour garantir la bonne exécution des demandes. Par exemple, les appareils mobiles et les passerelles domestiques intelligentes sont des dispositifs à ressources limitées où les utilisateurs peuvent installer des applications provenant de sources différentes. Dans le cas des passerelles domestiques intelligentes, Éviter toute mauvaise conduite dans les applications est important parce que ces dispositifs contrôlent souvent un environnement physique occupé par des personnes.

Pour satisfaire à ces exigences, l'objectif est de rendre les applications et les environnements d'exécution conscient et capable de faire face à des ressources limitées. Ceci est important parce que souvent ces exigences émergent ensemble (par exemple, les smartphones sont des appareils à ressources limitées fournissant un environnement d'exécution ouvert). Quand une application inclut des fonctionnalités pour réagir et modifier son comportement suite à l’apparition d’événements liés aux ressources, on dit que l’application est «consciente des ressources». Un système logiciel nécessite un environnement d'exécution approprié pour fournir de telles caractéristiques. Aujourd'hui, les applications s'exécutant sur des environnements d'exécution gérés (MRTEs), tel que Java, font partis des systèmes qui peuvent bénéficier de cette «conscience des ressources». En effet, les MRTEs sont régulièrement utilisés pour mettre en œuvre les intergiciels, par exemple en utilisant OSGi, en raison de leur sécurité, flexibilité, et de la maturité de l'environnement de développement. Ces intergicield fournissent souvent des fonctionnalités de monde ouvert, telles que la possibilité d'ajouter de nouvelles fonctionnalités après le déploiement initial du système. Pour soutenir la capacité d'adaptation et de gestion demandée par une exécution dans un monde ouvert, il est possible d'utiliser des techniques de génie logiciel à base de composants (CBSE). Hélas, certains MRTEs, tels que Java, ont été conçus pour exécuter une seule application à la fois, de sorte qu'ils manquent d’outils pour la gestion des ressources à grain fin.

\textbf{Cette thèse aborde le problème de la programmation pour créer des systèmes «conscient des ressources» supporté par des environnements d'exécution adaptés.} En particulier, cette thèse vise à offrir un soutien efficace pour recueillir des données sur la consommation de ressources de calcul (par exemple, CPU, mémoire), ainsi que des mécanismes efficaces pour réserver des ressources pour des applications spécifiques. Malheureusement, les mécanismes actuels nécessaires pour permettre la programmation de ressources dépendent fortement de la technologie cible. Par exemple, réserver  de la mémoire pour un processus natif Unix est différent de réserver de la mémoire pour un bundle OSGi (le premier problème consiste à créer un espace d'adressage virtuel tandis que le deuxième requiert l'utilisation conjointe d'un allocateur de mémoire et un ramasse-miettes spécifiques). En conséquence, les solutions que nous discutons dans nos recherches sont principalement ciblées sur la gestion des ressources dans le cadre des MRTEs. En particulier, nous nous concentrons sur ce genre d'environnement d'exécution lorsque nous présentons les contributions de cette thèse.

\subsection*{Défis}

Dans les solutions existantes qui permettent de surveiller la consommation des ressources et de réserver des ressources dans les MRTEs, nous trouvons deux inconvénients importants. La lutte contre ces inconvénients, qui sont décrits ci-dessous, est l'objectif de cette thèse.

Les solutions pour la surveillance de la consommation des ressources et leur réservation imposent un impact important sur les performances à l'exécution des applications. En particulier, les mécanismes basés sur l'instrumentation, qui offrent une bonne précision, réduisent de manière significative la performance des applications. Bien que cette restriction n’impacte pas les mesures des ressources consommées, il empêche leur utilisation dans un environnement de production. En conséquence, les ingénieurs sont obligés de choisir entre deux solutions non satisfaisantes - soit des performances réduites avec une bonne précision ou des performances acceptable avec une faible précision - lorsque les applications nécessitent d’être  conscientes de la consommation et de la réservation des ressources.
Malgré l'utilisation répandue des MRTEs pour exécuter des applications basées sur les composants et autres abstractions, la création d'outils permettant de gérer finement les ressources pour ces abstractions est encore une tâche complexe. En effet, la création d'abstractions, comme des modèles de composants, est de plus en plus commune. Beaucoup d'outillage existe pour le faire, en particulier pour définir de nouveaux langages dédiés. En outre, bien souvent, ces abstractions ciblent les MRTEs comme technologies permettant l’exécution en raison de leur sécurité et de la maturité des environnements de développement. Cependant, ces nouvelles abstractions posent un défi aux développeurs quand ils s’intéressent à la surveillance des ressources parce que ces nouvelles abstractions ne sont pas toujours offertes avec des mécanismes de surveillance des ressources ainsi que des débogueurs personnalisés. En conséquence, les développeurs utilisent des outils traditionnels qui peuvent seulement faire face aux concepts classiques tels que des objets, des méthodes et des emplacements de mémoire, au lieu des concepts plus spécifiques. La raison pour cela est que la définition d’un outillage pour une abstraction spécifique est une tâche ardue qui doit être mise en balance avec le public limité d'une telle abstraction.

Les défis de cette recherche peuvent être résumés dans les questions de recherche suivantes. Ces questions se posent à partir de l'analyse des inconvénients des outils actuels dans les paragraphes précédents. Il est utile de rappeler que ces questions se rapportent aux MRTEs.

\begin{enumerate}
\renewcommand{\theenumi}{\textit{QR\arabic{enumi}}}

\item Comment pouvons-nous fournir un soutien portable et efficace pour la surveillance de la consommation de ressources ?\label{qr:qr1}
\item Comment pouvons-nous choisir les mécanismes a utilisés pour garantir la réservation de ressources tout en maintenant un surcout d’exécution faible pour chaque composant logiciel? \label{qr:qr2}
\item Comment pouvons-nous tirer profit de la connaissance de l'architecture des applications pour aider un mécanisme de gestion des ressources? \label{qr:qr3}
\item Comment pouvons-nous faciliter la définition et la mise en place d'outils de surveillance pour de nouvelles abstractions de logiciels? \label{qr:qr4}
\end{enumerate}

\subsection*{Contributions}
Les résultats de cette thèse forment trois contributions qui visent à réduire (1) le coût de calcul pour effectuer la gestion des ressources, et (2) la complexité de la création d'outils de contrôle des ressources. Deux d'entre elles ciblent exclusivement le problème de la réduction du coût de calcul pour effectuer la gestion des ressources tandis que la troisième vise également le problème de faciliter la construction d'outils de suivi de l’utilisation des ressources. Ces contributions sont brièvement décrites ci-dessous.

\textbf{Contribution: un cadre de surveillance des ressources optimiste qui réduit le coût de la collecte des données de consommation de ressources.}
Le suivi de la consommation des ressources est le fondement de la programmation pour les systèmes conscients de leurs ressources.
Dans cette recherche, une nouvelle approche construite sur l'idée d'un contrôle adaptatif est présenté. L'approche, à savoir Scapegoat, est fondée sur quatre principes: i) souvent des applications sont construites en utilisant des abstractions telles que les composants que nous pouvons utiliser pour identifier et isoler la consommation des ressources, ii) lorsque l'environnement d'exécution est en cours d'exécution sur la ressource, nous pouvons utiliser la surveillance légère optimiste et toujours être sûr que nous serons en mesure de détecter les défaillances potentielles dans le temps, iii) il est possible d'identifier rapidement le composant défectueux une fois qu’un échec potentiel est repéré, et iv) il existe des mécanismes de contrôle que nous pouvons réutiliser parce qu'ils sont échangeables à l'exécution et offrent différents compromis entre le surcoût d’exécution et la précision. Scapegoat a été mis en œuvre et évalué dans ce travail et les résultats montrent la faisabilité et l'efficacité. Cette contribution répond aux questions de recherche \ref{qr:qr1} et \ref{qr:qr3}.

\textbf{Contribution: une méthodologie pour sélectionner les le support d’exécution des composants au moment du déploiement afin d'effectuer la réservation de ressources.}
La réservation ressources pour des applications spécifiques est une autre préoccupation dans la programmation des systèmes conscients de leurs ressources. Dans cette recherche, nous avançons que la mise à disposition des capacités de réservation de ressources dans le cadre de l’utilisation de composants logiciels ne devrait pas seulement être considérée lors de la conception et la mise en œuvre du modèle de composant. Au lieu de cela, nous soutenons qu'il est intéressant d'utiliser un mécanisme retardé pour choisir la technique de réservation de ressources pour chaque composant, et ce choix peut être fait en regardant les besoins en ressources de chaque composant au moment du déploiement. En bref, nous suggérons que - si un modèle de composant vise à soutenir le déploiement de composants avec des contrats de garantie ressources - les besoins en ressources et les technologies disponibles devrait être des variables de décision pour déterminer comment lier des composants à des abstractions de niveau système au moment du déploiement. Dans ce travail, nous démontrons cette hypothèse à travers la mise en place d’un prototype nommé Squirrel pour montrer les bénéfices potentiels de cette méthodologie. Cette contribution est une réponse aux questions de recherche \ref{qr:qr2} et \ref{qr:qr3}.

\textbf{Dans cette thèse, une approche générative pour créer des profileurs de mémoire personnalisées pour des abstractions spécifiques à un domaine, tels que les DSLs et modèles de composants, est proposée.}
L'approche consiste essentiellement dans un langage pour définir des profileurs et un générateur de profileur qui cible les mécanismes d'exploration de la mémoire en utilisant la technologie JVMTI.
Le langage a été conçu avec des contraintes qui, même si elles réduisent son expressivité, permettent d’offrir des garanties sur le comportement et la performance des profileurs générés. Pour évaluer l'approche, nous avons comparé les profileurs générés avec cette approche, les profileurs écrits manuellement et des outils traditionnels. Les résultats montrent que les profileurs générés ont un comportement similaire à celui des solutions écrites manuellement et spécifiquement pour une abstraction donnée. Les questions de recherche \ref{qr:qr1} et \ref{qr:qr4} sont adressés par cette contribution.


