

\section{Related Work} \label{sec:related-works}
%\todo{Walter + inti + olivier}

%Lots of approaches~\cite{Oreizy:2008:RSA:1370175.1370181} demonstrate the benefits of component-based systems to support runtime adaptation in order to handle Quality of Service (QoS) reservation or QoS improvement.
The Squirrel approach is mainly related to i) component isolation for fine-grained resource management, ii) efficient communication between containers, and iii) efficient container initialisation. We discuss the related works for these topics. 

%\fancynote{Inti}{I am not writing the story. I am just listing related papers. That's what you asked me last time}
% a paragrapth to talk about resource monitoring, I-JVM, Charles University (Binder), Other, the paper by Kouter
	

\textbf{Component isolation for fine-grained resource management.}
%A first solution to support resource aware component management is to use monitoring. 
In the Java world, several approaches describe the use of monitoring to achieve resource-aware programming~\cite{Guidec:2003:JMP,Maurel:2012:AME:2304736.2304763,Moreau:2005:RAP}.
In \cite{binder_portable_2006,Hulaas:2008:PTL} the authors propose mechanisms based on bytecode instrumentation to monitor CPU and memory consumption.
%The instrumentation consists in adding calls to a resource monitoring agent during class loading.
Yet, instrumentation introduces high overhead and the monitoring framework contaminates the measurements
%contaminates the 
%Furthermore, the measurements are contaminated by the monitoring framework
because it is performed in the context of the monitored JVM.
%To avoid such a negative effect, the authors of \cite{Marek:2013:SRC} introduce ShadowVM. This framework delegates monitoring tasks to separate processes.
%However, the required communication also produces additional CPU overhead.
Geoffray et al.~\cite{Geoffray:2009:I-JVM} introduce a modification to the garbage collector to achieve lightweight memory accounting for OSGi containers.
%This modification has lower overhead than using instrumentation. 
Our work shows how we can automatically isolate components with resource contracts with low overhead through the execution of components on isolated system containers.

% about communication
\textbf{Efficient Inter-Container/Process communication}.
In~\cite{Unrau:708530}, the authors devise mechanisms for IPC based on building a queue in shared memory. As in our case, these solutions perform two data copies.
Nonetheless, the authors do not consider the problem of sending messages to multiple targets, which is a concern for us.
In Android, Binder is used to support IPC. 
Although it is a single-copy mechanism, its common usage in applications still requires two copies due to marshaling.
The limitations of the built-in serialization in Java are described by Bouchenak et al.~\cite{bouchenak2003techniques}.
Protocol buffers \cite{protobuf} provides an Interface Definition Language to generate RPC artifacts in different languages. 
Although the performance is good, it makes mandatory the use of an IDL, which is often a burden for engineers.
We propose to use a combination of shared-memory-based channels and a high-performance serialization library to marshal POJOs.


% about initialization time, talk also about VM
\textbf{Fast container initialization.} 
%Several approaches show the need to improve the initialisation time of containers or VMs. 
A goal of the MVM~\cite{czajkowski2012multitasking} is to provide fast initialization of new Java Applications.
By sharing the standard library among JVM instances, the authors reduce initialization time for new applications the bootstrap classes are already loaded.
However, the MVM is a non-trivial modification to a JVM and its current implementation does not support the latest version of the JDK~\cite{czajkowski2012multitasking}. 
Android \cite{Kalkov:2012:REA:2388936.2388955,Nagata:2013:MIA:2569436.2570296} uses a process, named Zygote, that loads the standard classes and is a sort of pre-initialized Dalvik VM.
Zygote receives demands to create copies of itself that mutate into Dalvik VMs. %executing applications.
%The approach leverages the \textit{fork} system-call to clone a process and it also uses copy-on-write to reduce memory consumption and initialization time. 
%On the contrary, 
We use an external tool, CRIU, to speed-up the creation of new JVMs because current JVM implementations do not provide built-in facilities for doing so.