\section{Reference Implementation} \label{sec:kevoree}
%\fixme{The beginning of this section needs work, it starts by saying we introduce kevoree, then introduces models@run.time, then kevoree.}

Squirrel's reference implementation exploits the modesl@run.time approach and provides resource-awareness capabilities to the Kevoree component framework~\cite{Fouquet:2012:DCM:2304736.2304759}.
%to support models@run.time.
%We begin with a brief overview of Kevoree.
%Built on top of dynamic component frameworks, models@run.time denotes model-driven approaches to tame the complexity of dynamic adaptation~\cite{Morin:2009:TDA:1555001.1555028}.
Models@run.time denotes model-driven approaches to tame the complexity of dynamic adaptation~\cite{Morin:2009:TDA:1555001.1555028}.
%It pushes the idea of reflection one step further by considering the reflection-layer as a model.
The ``model at runtime'' is a reflection model that can be decoupled from the application (for reasoning, validation, and simulation purposes) and then automatically resynchronized.
Models can manage not only the application's structural information (i.e., the architecture), but can also be populated with other information, such as runtime monitoring data.

Kevoree is a component framework for distributed systems that builds and maintains a structural model of the system, following the models@run.time paradigm.
%and builds a structural model of the distributed system.
Kevoree is mainly used because: (i) it presents a snapshot of the distributed system, and (ii) it provides a language to drive the reconfigurations. 
\textit{Component} and \textit{Channel} are two of the concepts used in Kevoree models.
The former represents software units that provide the business value.
The latter, with the same role as connectors, are in charge of inter-component communication.
Channels encapsulate communication semantics (e.g., synchronous, unicast).

\subsection{A resource-aware container for CPU and I/O reservation} \label{sec:cgroups-based-reservation}

Our implementation leverages resource-reservation mechanisms at the system-level to provide containers for CPU and I/O reservation.
More specifically, it maps the concept of \textit{container} onto a \textit{cgroup}.
Both \textit{containers} and \textit{Kevoree components} are hierarchical structures that are easy to map onto cgroups' hierarchy.
Indeed, a container deploys components and a component runs threads.
To configure the container we setup a hierarchy of cgroups
using the following rules:
\begin{enumerate}
\item
%A fixed amount of resources are reserved for the Kevoree framework, which includes the components of the managed system. This serves as the upper resource-limit.
%To implement this rule we execute the Kevoree framework under a cgroup, while unmanaged applications are expected to run under other cgroups for isolation.
The Kevoree framework is started under a cgroup, using a fixed amount of resources that will be divided among the system's components. 
% that determines the total resource pool. % are reserved (for both).
%This value serves as the pool of resources the framework may use to deploy components.
%On the one hand, it is the maximum amount of resource the container will use when all the threads inside are demanding.
%On the other hand, it is the minimum amount it will receive when other applications outside the container are demanding.


\item
%We create a new resource-aware container for each component, represented by a cgroup.
Each component gets a new resource-aware container, also represented by a cgroup.
The component's contract is translated into a slice $S_c$ of the initial resource allotment, and the result is passed to
%configuration parameters to the cgroup.
the cgroup as configuration parameters.
A slice represents the
%minimum and maximum quantity of 
resources the component has available.
%This means that, although we support resource reservation, we are not supporting overcommitment.
%Since the scheduling unit for cgroups is a thread, we assign threads to cgroups in order to enforce resource reservation.
%However, we cannot assign threads directly to their component's cgroup because then all of them will obtain the same slice $S_c$.
%What we want is to share $S_c$ among all the threads within a component.
%Therefore, a \textit{hidden cgroup} is created that inherits the slice $S_c$ and provides a share to any thread assigned to it by using  \textit{Relative Shares Tunable Parameters}.

\item Since the scheduling unit for cgroups is a thread, we assign the component's threads to the cgroup to enforce resource reservation.

%\item Threads are assigned to the \textit{cgroup} created for the thread owner's component.
\end{enumerate}


\begin{figure}
\centering
%\usetikzlibrary{shadows,arrows}

% Define block styles  
\tikzstyle{cgroup_sty}=[draw, fill=white!20, text width=5em, text centered,
  minimum height=1.4em, rounded corners]

\tikzstyle{thread_sty}=[draw, thin, fill=yellow!20, text width=3em, text centered,
  minimum height=1.4em,dashed]

\tikzstyle{cgroup_container_sty}=[cgroup_sty,fill=gray!20]

\tikzstyle{cgroup_container_hidden_sty}=[cgroup_container_sty,dotted]


\tikzstyle{line} = [draw, thick, color=black!80, -latex']

% commands
\newcommand{\cgroup}[2]{node (p#1) [cgroup_sty]
  {{\scriptsize{#2}}}}

\newcommand{\cgroupContainer}[2]{node (p#1) [cgroup_container_sty]
  {{\scriptsize{#2}}}}

\newcommand{\hiddenCgroup}[2]{node (p#1) [cgroup_container_hidden_sty]
  {{\scriptsize{#2}}}}

\newcommand{\thread}[2]{node (t#1) [thread_sty]
  {{\scriptsize{#2}}}}

\begin{tikzpicture}[scale=0.7,transform shape]
\path \cgroup {1} {Root CGroup};
\path(p1.south) + (-2,-1.2) \cgroupContainer {2} {Cgroup for Kevoree};
\path(p1.south) + (1.7,-1.2) \cgroup {3} {Cgroup for other apps};

\path(p2.south) + (2.3,-1.2) \cgroupContainer {4} {Cgroup for Component2};
\path(p2.south) + (-1.5,-1.6) \cgroupContainer {5} {Cgroup for Component1};

%\path(p4.south) + (0,-1.0) \hiddenCgroup {6} {Hidden Cgroup};
%\path(p5.south) + (0,-1.0) \hiddenCgroup {7} {Hidden Cgroup};

\path [line] (p1) --  node [left] {50\%} (p2);
\path [line] (p1) --  node [right] {50\%} (p3);

\path [line] (p2) --  node [left] {45\%} (p4);
\path [line] (p2) --  node [left] {10\%} (p5);
%\path [line] (p5) --  node [left] {100\%} (p7);
%\path [line] (p4) --  node [right] {100\%} (p6);

\path(p5.south) + (-1.5,-1.2) \thread {1} {Thread1};
\path(p5.south) + (1.8,-1.0) \thread {2} {Thread3};
\path(p5.south) + (0,-1.7) \thread {3} {Thread2};

\path(p4.south) + (0,-1.2) \thread {4} {Thread4};

\path [line] (p5) --  node [above,left] {33\%} (t1);
\path [line] (p5) --  node [above,right] {33\%} (t2);
\path [line] (p5) --  node [left] {33\%} (t3);

\path [line] (p4) --  node [right] {100\%} (t4);

\end{tikzpicture}
\caption{Reserving CPU by mapping components to cgroups} \label{fig:Mapping}
\vspace{-0.5cm}
\end{figure}

This scheme is used for CPU, I/O throughput and network bandwidth.
Each type of resource requires a different \textit{container type}.
% since configuring the container depends on varying properties defined on each cgroup subsystem.
Figure \ref{fig:Mapping} shows an example 
%of how our implementation uses 
using cgroups to reserve CPU for a system with two components.
In the tree, every edge is labeled with the cgroup's CPU slice.
% the cgroup obtains from its parent.
A slice is set for Kevoree, while unmanaged apps are maintained in separate containers.
Applying rule 2, CPU slices are assigned to \textit{Component 1} and \textit{Component 2} according to their resource contracts.
Following rule 3, every thread in \textit{Component1} receives $ 33\% $ of the component's slice.
%When configuring the containers, we apply rule 2 and assign \textit{Component 1} and \textit{Component 2} CPU slices according to their resource contracts.
%Finally, slices for Kevoree and for the rest of the system are assigned.
%In our case, we selected as description the number of instructions the container is able to execute because, as we show in section \ref{contracts}, our contracts use number of instructions in their definition. 
%For instance, if we assume that our container is able to execute 1002000012 instructions per seconds and we use the contracts in listing \ref{} then we obtain slices as in Figure \ref{Mapping}.
%\todo{Inti: create contracts for 2 components, 102345600 and 456789010 instructions}
%Finally, slices for Kevoree and for the rest of the system are assigned.
%This requires taking into account the trade-off between resource for components in Kevoree and for applications outside Kevoree.

%Assigning little resources to Kevoree allows external applications to execute smoothly but the number of admitted components in Kevoree decreases.
%On the contrary, if Kevoree uses a big slice then the performance of other applications is affected.

\subsection{Containers for Memory reservation} \label{overview-memory-reservation}

Memory reservation poses a unique challenge.
% for the usage of cgroups from within the Java runtime.
Although there is a cgroup-hierarchy for memory, it is not well suited for shared memory because the subsystem cannot precisely account for the consumption of each thread.
As a result, if we use cgroups to deal with memory, accounting would depend on the behavior of the garbage collector, which is hard to predict.
That makes cgroups inadequate to check or enforce component contracts in a single JVM process.
We have devised two mechanisms
%to represent containers for memory reservation.
to serve as memory containers.
In the first mechanism, all containers exist in a single process and resource limits are enforced by leveraging previous approaches
%to resource monitoring for Java (e.g., using bytecode instrumentation to account for consumption).
that use bytecode instrumentation to account for consumption.
The second mechanism
%delegates reservation to a lower level in the software stack by isolating
isolates components into new processes and then uses cgroups. % for resource management.
The rest of the section describes both solutions.

\subsubsection{Memory management based on monitoring.}
A memory reservation container ensures that its components have access to the memory they require.
Memory requests should only fail if a component violates its contract.
Implementing such a container is simple if memory monitoring is available at the application-level and memory requests are intercepted.
%DO WE NEED THIS PART?
%A container is a service that receives notifications when a component is violating its contract and then takes action to forbid future memory requests until a proper utilization level is reached.   
We use Scapegoat \cite{gonzalezherrera:hal-00983045} for memory consumption monitoring by defining multiple memory-aware containers within a single JVM.
In short, each container registers its component in ScapeGoat and receives a notification if a component violates its contract.
%This has an impact on the application's performance 
This introduces CPU and memory overhead for each component because of the instrumentation code.
%the instrumentation code imposes penalties on CPU/memory during the execution time of each component.
The main advantages are portability and simplicity.

\subsubsection{Isolation of components in separate processes.}
The second approach maps each container onto a separate process.
Reservation is achieved using cgroups as described in section \ref{sec:cgroups-based-reservation}.\footnote{In practice, we use cgroups to reserve memory, but we also bound the Java Heap to limit the consumption in Java code.}
%
%Reducing deployment time and complexity is critical since we deal with continuous deployment; hence we aim at automating the process of component isolation.
To do so, we start from an extended deployment model as shown in figure \ref{fig:Overview}.
%A transformation follows the rules below is applied to the model:
The model is then transformed using the following rules:
\begin{enumerate}
\item  Component isolation: each \textit{set of components} with a shared memory contract is isolated in a separate \textit{JVM node} within the same \textit{physical device}.
%A contract regarding memory reservation is shared if the involved components reside in the same \textit{JVM node} in the source model.
\label{rule1-model-transformation} 

%\item Channel adjustment: Every channel that connects two components in the source model that is isolated in the target model (by the application of rule \ref{rule1-model-transformation}) is adjusted to reflect the semantics of the source model. This includes changing the channel type as well as modifying some of its properties.
\item Channel adjustment: channels that connect isolated components are updated to reflect the semantics of the source model. This includes changing the channel type and modifying some of its properties.
\end{enumerate}
The resulting model can be deployed.
%in the newly created Kevoree instances through a dedicated \textit{Kevoree Group}.

%Figure~\ref{model_transformation} shows how to transform the model of a simple application.
%In this case, we assume that every component has its own contract regarding memory reservation.
%Following the rules, we only need to create an additional \textit{JVM node} and to adjust the \textit{channel} that connects components \textit{Video Recorder} and \textit{Video Coder}.  

%\begin{figure*}
%\centering
%\usetikzlibrary{shadows,arrows}
\pgfdeclarelayer{background}
\pgfdeclarelayer{foreground}
\pgfsetlayers{background,main,foreground}

% Define block styles  
\tikzstyle{component_sty}=[draw, fill=white!20, text width=5em, text centered,
  minimum height=1.4em, rounded corners]

\tikzstyle{deployment_node_sty}=[draw, thin, text width=2.5em, text centered,
  minimum height=1.4em]

\tikzstyle{physical_node_sty}=[deployment_node_sty, fill=yellow!20, text width=2.5em, dashed]

\tikzstyle{virtual_node_sty}=[deployment_node_sty, fill=gray!30, dotted]

\tikzstyle{line} = [draw, thick, color=black!80, -latex']

\tikzstyle{channel_sty}=[draw, fill=green!20, text width=5em, text centered,
  minimum height=1.4em, dashed]


% commands
\newcommand{\component}[2]{node (p#1) [component_sty]
  {{\scriptsize{#2}}}}

\newcommand{\virtualNode}[2]{node (p#1) [virtual_node_sty]
  {{\scriptsize{#2}}}}



\begin{tikzpicture}[scale=0.9,transform shape]

\begin{pgfonlayer}{foreground}
	\path \component{1} {Video Recorder};
	\path (p1.south) + (0,1) \component{2} {Video coder};
	\path (p1.east) + (2.2,0.0) \component{3} {Video storage};
\end{pgfonlayer}

%\path \virtualNode{1} {node0};

% virtual node 0
\path (p2.west)+(-0.16,0.6) node (g) {};
\path (p1.east)+(0.16,-0.8) node (h) {};
\path[virtual_node_sty] (g) rectangle (h) ;
%\path (h) +(-1.3,0.2) node (asrs) {java node 0};
\begin{pgfonlayer}{background}
% physical node 0
        \path (g.west)+(-0.12,0.24) node (g) {};
        \path (h.east)+(0.12,-0.24) node (h) {};
        \path[physical_node_sty] (g) rectangle (h);
     	\path (h) +(-1.6,-0.2) node (asrs) {device 0};         
\end{pgfonlayer}

% virtual node 1
\path (p3.west)+(-0.16,0.6) node (g) {};
\path (p3.east)+(0.16,-0.8) node (h) {};
\path[virtual_node_sty] (g) rectangle (h) ;
\path (h) +(-1.3,0.2) node (asrs) {java node 1};
\begin{pgfonlayer}{background}
% physical node 1
        \path (g.west)+(-0.12,0.24) node (g) {};
        \path (h.east)+(0.12,-0.24) node (h) {};
        \path[physical_node_sty] (g) rectangle (h);
     	\path (h) +(-1.6,-0.2) node (asrs) {device 1};         
\end{pgfonlayer}

%channels
\path (p2) + (-2.5,-0.2) node (channel1) {};
\path [line] (p1.west) -- (channel1) ;
\path [line] (channel1) --  (p2.west);
\path[channel_sty] (channel1) circle (0.4);

\path (p1) + (2.3,2.1) node (channel2) {};
\path [line] (p2.north) -- (channel2) -- (p3.north);
\path[channel_sty] (channel2) circle (0.4);

\path (h) node (temp1) {};

%AFTER MODEL TRANSFORMATION

\path (temp1.east) + (0.1,4) node(l0) {}; 
\path (temp1.east) + (0.1,-1) node(l1) {};
\path[draw, thick, color=black!80, dashed] (l0) -- (l1);

\begin{pgfonlayer}{foreground}
	\path (temp1.east) + (2,1) \component{1} {Video Recorder};
	\path (p1.east) + (1.6,0) \component{2} {Video coder};
	\path (p2.east) + (2.2,0) \component{3} {Video storage};
\end{pgfonlayer}

% virtual node 0
\path (p1.west)+(-0.16,0.6) node (g) {};
\path (p1.east)+(0.16,-0.8) node (h) {};
\path[virtual_node_sty] (g) rectangle (h) ;
\path (h) +(-1.3,0.2) node (asrs) {java node 0};
% virtual node 1
\path (g) node (gTemp) {};
\path (p2.west)+(-0.16,0.6) node (g) {};
\path (p2.east)+(0.16,-0.8) node (h) {};
\path[virtual_node_sty] (g) rectangle (h) ;
\path (h) +(-1.3,0.2) node (asrs) {java node 1};
\begin{pgfonlayer}{background}
% physical node 0
        \path (gTemp.west)+(-0.12,0.24) node (g) {};
        \path (h.east)+(0.12,-0.24) node (h) {};
        \path[physical_node_sty] (g) rectangle (h);
     	\path (h) +(-3,-0.2) node (asrs) {device 0};         
\end{pgfonlayer}

% virtual node 1
\path (p3.west)+(-0.16,0.6) node (g) {};
\path (p3.east)+(0.16,-0.8) node (h) {};
\path[virtual_node_sty] (g) rectangle (h) ;
\path (h) +(-1.3,0.2) node (asrs) {java node 2};
\begin{pgfonlayer}{background}
% physical node 1
        \path (g.west)+(-0.12,0.24) node (g) {};
        \path (h.east)+(0.12,-0.24) node (h) {};
        \path[physical_node_sty] (g) rectangle (h);
     	\path (h) +(-1.6,-0.2) node (asrs) {device 1};         
\end{pgfonlayer}

%channels
\path (p2) + (-1.6,2.1) node (channel1) {};
\path (p2.north)+(-0.7,-0.2) node (tt0) {};
\path [line] (p1.north) -- (channel1) -- (tt0);
\path[channel_sty] (channel1) circle (0.4);

\path (p3) + (-1.6,2.1) node (channel2) {};
\path [line] (p2.north) + (0.7,0) -- (channel2) -- (p3.north);
\path[channel_sty] (channel2) circle (0.4);


\path (h) node (temp1) {};

% legend
% physical nodes
\path (temp1)+(1.0,2.8) node (h) {};
\path (h)+(0.8,0.8) node (g) {};
\path[physical_node_sty] (g) rectangle (h);
\path[right,text width=4em] (g) +(0,-0.4) node (asrs) {physical devices};

% virtual nodes
\path (h)+(0,-0.9) node (h) {};
\path (g)+(0,-0.9) node (g) {};
\path[virtual_node_sty] (g) rectangle (h);
\path[right] (g) + (0.0,-0.4) node (asrs) {virtual nodes};

% components
\path (h)+(0,-0.9) node (h) {};
\path (g)+(0,-0.9) node (g) {};
\path[component_sty] (g) rectangle (h);
\path[right] (g) +(0,-0.4) node (asrs) {components};

% channels
\path (g)+(-0.4,-1.35) node (g) {};
\path[channel_sty] (g) circle (0.4);
\path[right] (g) +(0.38,-0.0) node (asrs) {channels};

\end{tikzpicture}
%\caption{Transforming deployment model to achieve memory reservation} \label{model_transformation}
%\end{figure*}

\paragraph{Runtime initialization through cloning.}
The approach to memory reservation based on isolation deploys each \textit{set of components} into separate processes.
This involves two steps: creating new instances of the runtime, and deploying a \textit{set of components} on each instance.
To reduce deployment time, instead of starting processes from scratch,
%we create clones of a base runtime for new instances.
new instances are cloned from a base runtime.
The base runtime is created offline. %, is used to deploy components through fast cloning and configuration.
Both steps, base runtime creation and cloning, are based on CRIU.\footnote{\url{criu.org}}
This tool allows snapshoting a process and starting any number of clones from the snapshot.
In essence this \textit{forks} the process.
%creating in practice a sort of \textit{forking} that is hard to perform otherwise.
%Afterwards, the cloned runtime is adapted by deploying a new model that includes an application's component.

%The implementation includes two classes of Kevoree instances, the management instance class $R_M$ and the base-instance class $R_b$.
%The former is a frond-end that interacts with external users and is in charge of: 1) modifying the model to support isolated deployment as section \ref{overview-memory-reservation} describes, 2) creating clones based on $R_b$ and, 3) sending messages to clones with the aim of deploying components.
%The $R_b$ class of instance executes a component that implements the system's adaptation by waiting for a message that includes the new model to deploy.

\paragraph{Channel for intra-node communication.}
Channels are meant to send arbitrary POJO structures from one component to another.
When components are isolated into separate processes,
%a channel transforms a POJO structure twice:
a channel must marshal and unmarshal the POJO using a representation suitable for inter-process communication.
%i) marshal the POJO into a representation suitable for IPC in the source and, ii) unmarshaling it in the destination.
In practice, a channel must copy data at least twice, no matter what IPC mechanism is used.
%This fact has an impact on the class of IPC mechanism we can leverage.

In this chapter we propose a new channel for intra-node communication.
Communication is performed through a message queue built on top of shared memory using an alternative high-performance framework to serialize objects.
Each channel is mapped to a shared-memory region that hosts a synchronized queue of messages.
This region contains three sections: an array of blocks to store message chunks, 
a set of free blocks, and a circular queue in which an element points to a list of chunks.
We use the procedure described in \cite{Unrau:708530} to synchronize senders and receivers, but we also support broadcast semantics without unnecessary additional copies.
Our approach copies the POJO from the sender's heap to shared-memory during data marshaling, then every receiver makes a copy from shared-memory.
The implementation uses a high-performance serialization framework\footnote{\url{https://github.com/RuedigerMoeller/fast-serialization}}
% ~\cite{Moeller:2014:Online} 
instead of Java's built-in serialization mechanism
and is able to serialize arbitrary objects with better performance.
%by using alternative binary representation and its own cached reflection mechanism.
