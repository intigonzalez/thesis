%\paragraph{Practical considerations \todo{Move to perspectives}}
%A component does not execute on top of a flat operating system process but on top of a managed runtime environment, the JVM.
%As a consequence, we devote resources to the component but also to the middleware.
%Due to the presence of the garbage collector, we do not know the exact amount of memory new \textit{JVM nodes} should be assigned with.
%In fact, we can only state that the reserved memory must equal the component reservation plus the middleware requirement plus any amount that allows an efficient behavior %of the garbage collector.
%Chances are that any prediction will overestimate the real consumption, leading to the problem of wasting resources due to internal fragmentation.

\section{Conclusion} \label{sec:conclusions}

%\enlargethispage{0.3cm}

This chapter presents Squirrel, a framework that provides resource management capabilities to dynamic component-based frameworks.
Squirrel proposes choosing component-to-system mappings at deployment time for better resource management.
This strategy is performed automatically by checking the resource availability and transforming the application's structure to run the application on resource-aware containers.
Containers describe how to map components to system abstractions
%in ways that reduce the cost of resource management.
allowing for different trade-offs in resource management.

We present an implementation of Squirrel that manages CPU, I/O and memory, and provide performance analyses and a comparison of different design decisions.
The experiments
%to assess our implementation's behavior 
show that choosing the right component-to-system mappings at deployment-time reduces CPU overhead and/or memory use.
They also highlight that optimizing mappings is essential to reducing isolation and communication overhead to acceptable levels.
%and make it affordable.

