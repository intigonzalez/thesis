%------------------------------%
\selectlanguage{english}
\chapter{Abstraction and resource management}
\label{chap:abstractions_and_resource_management}
\markboth{Abstraction and resource management}{Chapter2}
%------------------------------%

\coolphrase {Hey, look at this}{Inti Gonzalez-Herrera}

\section{CBSE and Resource Aware}


Discutir aqui que problemas nuevos surgen cuando se quiere hacer resource management en component-basedsystems. Por ejemplo, a quien se le asigna el consumo.

To be specific, for the purposes of this book, software components
are executable units of independent production, acquisition, and deployment
that can be composed into a functioning system. To enable composition, a
software component adheres to a particular component model and targets a
particular component platform.

By enforcing a strict separation between interface and implementation and by making software
architecture explicit, component-based programming can facilitate the implementation and
maintenance of complex software systems [1]. Indeed, these two principles form the basis for two
essential properties: adaptability and manageability. Their role as units of software deployment and
configuration in particular, are well understood: they allow for pre-runtime adaptation in order to
suit arbitrary deployment environments (construction of dedicated software infrastructures), evolution
in requirements and technical evolution (maintenance), and organizational evolution (integration,
interoperation). When seen as runtime structures, components can serve as the basis for software
reconfiguration. By fully delineating subsystem boundaries, they provide a natural scope for
reconfiguration actions and a natural target for system instrumentation and supervision. Coupled with
the use of meta-programming techniques, component-based programming can hide from application
programmers some of the complexity inherent in the handling of non-functional aspects in a software
system, such as distribution and fault tolerance, as exemplified by the container concept in Enterprise
Java Beans (EJB), CORBA Component Model (CCM), or Microsoft .Net

It is likely that components of different qualities (level of performance, resource efficiency,
robustness, degree of certification, and so on) will be available at different
prices.

\section{Query languages about memory management}

\begin{enumerate}
\item DiSL: a domain-specific language for bytecode instrumentation, Marek, Villazon, Zheng, Ansaloni, Binder, Qi, AOSD'12

\item Flexible and efficient profiling with aspect-oriented programming, Binder, Ansaloni, Villazón and Moret, CPE, 2011

\item A portable and customizable profiling framework for Java based on bytecode instruction counting, Binder, PLAS'05

\item Profiling with AspectJ, Pearce, Webster, Berry and Kelly, SPE, 2007.

\item Controlled dynamic performance analysis, Reiss, WOSP'08

\item A meta-aspect protocol for developing dynamic analyses M Achenbach, K Ostermann, RV'10

\item Comprehensive Profiling Support in the Java Virtual Machine. S Liang, D Viswanathan, COOTS'99

\item Profiling Field Initialisation in Java, Nelson, Pearce, Noble, RV'13.

\item A dynamic optimization framework for a Java just-in-time compiler, Toshio Suganuma, Yasue, Kawahito, Komatsu, Nakatani, OOSPLA'01.
\end{enumerate}



\section{Languages to deal with resource aware}
