\selectlanguage{english}
\chapter{Building Efficient Domain-Specific Memory Profilers}
\label{chp:dsl-memory}
\markboth{Building Efficient Domain-Specific Memory Profilers}{Chapter6}

\coolphrase{
\textit{Jack (V.O.)} -- Babies don't sleep this well.\hspace*{4.9cm} \\
\vspace{1 mm}

\textbf{(Jack's bedroom -- night -- Jack lies sound asleep)}\\
\vspace{1 mm}

\textit{Jack (V.O.)} -- I became addicted [\dots]\hspace*{5.7cm} \\
\vspace{1 mm}

\textit{Jack (V.O.)} -- If I didn't say anything, people assumed the worst.  They  \\
                        cried harder. I cried harder [\dots]\hspace*{4.1cm} \\
\vspace{1 mm}
                  
\textit{Jack (V.O.)} -- [\dots] the guys with cancer [\dots] ``Free and Clear'',\hspace*{1.3cm} \\
                        my  blood parasites group Thursdays [\dots]\hspace*{2.5cm} \\
                        ``Seize The Day'', my tuberculosis Friday night. \hspace*{1.5cm}
}{Fight Club}

\lstset{
  aboveskip=5mm,
  belowskip=5mm,
  showstringspaces=false,
  columns=flexible,
  basicstyle=\color{black}\scriptsize,
  numbers=none,
  numberstyle=\color{gray}\scriptsize,
  keywordstyle=\color{black}\scriptsize\bfseries,
  commentstyle=\color{dkgreen}\scriptsize,
  stringstyle=\color{purple}\scriptsize,
  identifierstyle=\color{dkgray}\scriptsize,
  breaklines=true,
  breakatwhitespace=true,
  tabsize=4,
  captionpos=b,
}

\lstdefinelanguage{AlgLang}
{
literate={aaa}{bbb}3,
morekeywords={input, foreach, action, if, return, routine, values},
morekeywords={instances_for, have_id},
sensitive=true,
%frame=tblr,
morecomment=[l]{//},
morecomment=[s]{/*}{*/},
morestring=[b]",
}

\lstdefinelanguage{DSL}
{
literate={aaa}{bbb}3,
morekeywords={data, set_type, void, int, bool, Objects, THIS, is, ENTITY},
morekeywords={instances_for, have_names},
sensitive=true,
morecomment=[l]{//},
morecomment=[s]{/*}{*/},
morestring=[b]",
}

\lstdefinelanguage{OCL1}
{
literate={aaa}{bbb}3,
morekeywords={context,inv,and,or,self, if, not, else, endif, then},
sensitive=true,
frame=L,
morecomment=[l]{//},
morecomment=[s]{/*}{*/},
morestring=[b]",
xleftmargin=1\parindent,
}

\lstdefinelanguage{DSL2}
{
literate={aaa}{bbb}3,
frame=L, % tblr
%numbersep=2pt,
numbers=left,
numberstyle=\color{black}\scriptsize,
morekeywords={void, int, bool, in, is, or, and, struct, tableOf, ret},
morekeywords={membership, initialValues, updates, structure, initialObjects, create, using, constructor, foreach},
sensitive=true,
morecomment=[l]{//},
morecomment=[s]{/*}{*/},
morestring=[b]",
ndkeywords={false, true, this, referrer, threads, this_structure},
ndkeywordstyle=\color{blue}\bfseries,
xleftmargin=2\parindent
}

\lstdefinelanguage{OQL}
{
literate={aaa}{bbb}3,
numbers=left,
numberstyle=\color{black}\scriptsize,
morekeywords={SELECT, FROM, WHERE, UNION, AS, DISTINCT, ALL, GROUP, BY},
sensitive=true,
frame=tblr,
morecomment=[l]{//},
morecomment=[s]{/*}{*/},
morestring=[b]",
xleftmargin=1\parindent
}

\lstdefinelanguage{CYPHER}
{
literate={aaa}{bbb}3,
morekeywords={MATCH, RETURN, WITH, sum},
sensitive=true,
frame=tblr,
morecomment=[l]{//},
morecomment=[s]{/*}{*/},
morestring=[b]",
numbers=left,
numberstyle=\color{black}\scriptsize,
xleftmargin=1\parindent
}



The \gls{SLE} community aims at reducing the effort required in engineering new languages and their corresponding development tools, thus improving the efficiency of both people in charge of designing new languages and their users~\cite{sle}. 
However, as far as we know, they do not take into account profiling tools, which are essentials for software maintenance and optimization.
Indeed, although specific tools are needed to monitor running systems in order to detect defects or abnormal behaviors~\cite{duesterwald2000software, Jovic:2011:CMY:2076021.2048081},
little support exists to ease their creation.

In this chapter, we focus on the problem of easing the creation of memory profilers for domain-specific software abstractions that are designed to be executed on top of MRTEs. 
We first propose a metalanguage to specific what data about the memory use is of interest in a domain (see Section~\ref{sec:approach}).
A profiler is then generated to collect the data and present it in terms of concepts of that language. 
In addition, we present a tooled DSL based on such a metalanguage, which generates profilers for the JVM (see Section~\ref{sec:dsl-implementation}).
An important point of our approach is the low overhead induced by these profilers; this makes them usable in production environments (see Sections~\ref{sec:expressiveness} and~\ref{sec:dsl-evaluation}).

The contributions of this chapter are as follows:
\begin{itemize}
\item A metalanguage to describe what information a profiler must collect.
In addition, programs in this metalanguage also defines how to collect the information.
Although knowledge of the underline execution model is required, the procedure to obtain data is mostly defined without using low-level details.  

\item A concrete implementation of this metalanguage that target the JVM.
In particular, by using the \glslink{JVMTI}{JVMTI}, we are able to generate memory profilers with low overhead.
Concrete profilers already generated are portable to any implementation of the JVM that supports JVMTI.

\item A discussion of the metalanguage's expressiveness, and an evaluation of the performance overhead induced by three profilers in real-world use cases.
\end{itemize}

\section{Approach}\label{sec:approach}

In this chapter, we propose a tool to create custom memory profilers for MRTEs.
We are interested on easing the task of defining new profilers without sacrificing their performance regarding CPU consumption.
Keeping the profiler's overhead as low as possible is of utmost importance for us because lightweight profilers can be use both during the development phase and during the application execution in a production environment.
To reach this goal, we propose a Domain Specific Language (DSL) and its code generator which aims at describing and generating efficient online memory profilers. 

We next present the syntax, semantics and usage examples of our domain-specific language.


\subsection{Brief overview of the domain}

Memory profilers aim at capturing information regarding how an application use memory.
In an object-oriented runtime environment such as Java, this information can be as simple as the number of objects of a specific class, but it can also be  as complex  as the list of possible memory leak sources.
Along this paper, the term memory profiler refers to any kind of process to retrieve data about the memory usage.
Some examples include: computing the number of objects reachable from a specific class object; finding out if there is an instance of class $A$ which is referencing an instance of class $B$.
A last example could be computing for each instance of the class \textit{String} its length and the number of references to it.
It is worth mentioning that the data collected by a profiler may have an arbitrary type.
For instance, in the previous examples the types are primitives integer, boolean and a non-primitive type.

%The rest of this section introduces the vocabulary we use in the domain of memory profilers.

In this paper, we are interested in \textbf{\textit{objects}} as in object-oriented programming.
We also see an \textit{object} as an atomic entity that consumes memory to store the values of its attributes.
We have  reduced the available operations on objects to: accessing attributes, obtaining the amount of memory used to represent the object, and accessing meta-data such as the class name.

The \textbf{\textit{memory heap}} is the region of memory used to store dynamically allocated \textit{objects}.
Although this concept is pervasive in general purposes programming languages, we are only interested on MREs such as Java where every \textit{object} must be allocated in this region.

A \textbf{\textit{structure}} is an important concept in our domain.
It consists of a set of related \textit{objects} in the \textit{memory heap}.
The smallest non-empty \textit{structure} we can consider, is a structure containing a single object.
The \textit{memory heap} is the universe of objects and each \textit{structure} is a subset of this universe.

A \textbf{\textit{memory profile}} is a value associated to a \textit{structure} which can be derived out of the indivdual \textit{objects} included in the \textit{structure}. 
An example of useful derived value for a \textit{structure} is its total size: $\textit{total\_size(S)} = \sum_{o \in S} {sizeof(o)}$.
A common usage is to identify many \textit{structures} in the heap to obtain a \textit{memory profile} for each of them.
These two steps: identifying \textit{structures} and computing their \textit{memory profile} correspond to what we call \textbf{\textit{memory profiling}}. 

Finally, a \textbf{\textit{structure type}} provides a description of the behavior of a set of \textit{structures}. 
In particular, it provides (i) a function to evaluate whether an object is member of the \textit{structure}, (ii) a way to define the values corresponding to the \textit{memory profile} of the \textit{structure}, and finally (iii) a factory to identify all  corresponding \textit{structures} in the \textit{memory heap}.

\subsection{Abstract Syntax}\label{sec:abstract-syntax}

The metamodel shown in Figure~\ref{fig:as} describes the abstract syntax of our DSL.
The main concept of this metamodel is a \textit{CustomProfiler} which is composed of \textit{UserDefined} types and a \textit{StructureFactory}.
The concepts related to \textit{UserDefined} types are shown on the left part of the metamodel, while the right part describes the \textit{StructureFactory} which represents both the set of  structures to identify and the value to compute on these structures.

\subsubsection{User-defined types}
In addition to various \textit{BasicTypes} such as \textit{Integer}, \textit{String} and \textit{Boolean}, the language supports the definition of both \textit{Records} and \textit{Lists}.
As expected, a \textit{record} contains \textit{fields} to hold values of previously defined types.
Likewise, a \textit{list} refer to a \textit{base type}; hence all the members of a list must be of the same type.
In a custom profiler, \textit{UserDefined} types can be composed in arbitrary ways as long as no type contains a recursive declaration.
We can formalize such a constrain using OCL (Object Constraint Language~\footnote{http://www.omg.org/spec/OCL/}):

\begin{lstlisting}[escapeinside={(*}{*)},
label=fig:membership,
language=OCL, frame=tblr]
context Record inv: 
   not fields->oclAsSet()->closure(t)->exists(t | t = self)

context List inv:
   not baseType->oclAsSet()->closure(t)->exists(t | t = self)
\end{lstlisting} 

A \textit{List} has operations to manipulate any value which represent a list.
Figure~\ref{fig:as} shows a subset of these operations.
In general, these operations correspond to the set of \textit{standard} operations of any implementation of the list data type.

\subsubsection{Defining structures to profile}
Defining the \textit{StructureFactory} is the core of writing a custom profiler.
A \textit{StructureFactory} contains an \textit{Expression} through the \textit{instances} relationship which indicates a pattern to identify structures in the memory heap.
Notice that a single instance of \textit{StructureFactory} creates many data structures in memory; thus the \textit{Expression} corresponds to a list indicating that a new structure must be instantiated for each element of the list.
Forcing the \textit{Expression} to be a \textit{List} can be easily formalized using OCL:

\begin{lstlisting}[escapeinside={(*}{*)}, label=fig:instances, language=OCL, frame=tblr]
context StructureFactory inv: instances.type.oclIsTypeOf(List)
\end{lstlisting}

Defining a new \textit{StructureFactory} implies defining its \textit{type} which is an instance of \textit{StructureType}.
This concept describes the mechanism used to populate, out of objects, a structure and its information.
In short, each structure in memory instanciated through a \textit{StructureFactory} has a type \textit{StructureType}.
For example, we may be interested in finding all the \textit{SimplyLinkedList} in the memory snapshot depicted in Figure~\ref{fig:simple_snapshot}.
In such a case, there are two \textit{SimplyLinkedList}, but we only need one mechanism to identify them because they have the same pattern in memory. 

\begin{figure*}
\centering
\includegraphics[width=0.87\linewidth]{chapter6/fig/AS}
\caption{Custom profiler Metamodel}
\label{fig:as}
\end{figure*}

\begin{figure}
\centering
\includegraphics[width=0.65\linewidth]{chapter6/fig/lists}
\caption{Memory snapshot with three linked lists}
\label{fig:simple_snapshot}
\end{figure}

A \textit{StructureType} is composed of \textit{Assignments} that are used as \textit{initialValues} for each \textit{Variable} holding information - similar to a constructor in object-oriented programming.
Once a new structure is created, the \textit{Assignments} are executed to assign the initial value of each \textit{Variable}.
Observe that \textit{variables} do not refer to a \textit{Type}.
Our DSL is strongly typed and the \textit{type} of each user-defined variable is inferred from its initial value.
Nevertheless, there is a built-in variable in each \textit{StructureType} that is only accessible during initialization.
Its type is predetermined as part of the language specification.
We force the use of the proper type using OCL:

\begin{lstlisting}[escapeinside={(*}{*)}, label=fig:instances, language=OCL, frame=tblr]
context StructureType inv:  initialValues->exists(a: Assignment | 
     a.lvalue.name = 'initialObjects' and a.rvalue.type.oclIsTypeOf(List)
   ) 
\end{lstlisting}

In addition, a \textit{StructureType} contains a boolean \textit{expression} which is the \textit{membership function} used to decide whether an object should be included in the structure instance.
Finally, it also contains a set of \textit{Assignments} to update the value of each \textit{variable} every time a new object is included in the structure.
This set of \textit{Assignments} is used to compute the actual value of the memory profile.
The major constraint regarding these \textit{updates} is that they must refer to already initialized \textit{variables} and the new assigned values must match the previous types.
We formalize such a constrain using OCL:

\begin{lstlisting}[escapeinside={(*}{*)}, label=fig:lvalue, language=OCL, frame=tblr]
context StructureType inv:  updates->forAll(a: Assignament | 
    self.initialValues->exists(aa : Assignament | 
       aa.lvalue = a.lvalue and aa.rvalue.type = a.rvalue.type
     ))
\end{lstlisting}

In our DSL, \textit{expressions} play a big role.
For the sake of readability, Figure~\ref{fig:as} only shows a couple of concepts related to them.
However, it is noteworthy that, in addition to \textit{arithmetic}, \textit{boolean} and \textit{literals} for basic types, the language includes lambda expressions, literal for records and lists.
Moreover, the language defines \textit{built-in rvalues} which are nothing but expressions initialized by the runtime within a specific scope.
Instead of being user-defined, the types of these expressions are also defined by the runtime.
There are two types of \textit{built-in rvalues}, target independent and dependent.
Among the firsts, we have the list of  \textit{objects}, a reference to the \textit{current data structure} and a reference to the \textit{current object}.
Target dependent \textit{rvalues} in Java include the list of \textit{loaded classes} and the list of \textit{threads}.
The precise meaning of these \textit{rvalues} as well as their scopes are precisely discussed in sections~\ref{sec:concrete-syntax}, ~\ref{sec:semantic} and~\ref{sec:implementation} along the concrete syntax, the language semantic and the tooling support.

Finally, if we use the memory snapshot depicted in Figure~\ref{fig:simple_snapshot}, calculating the number of nodes belonging to a specific \textit{SimplyLinkedList} is an example that illustrates the language's concepts.
To solve this problem we can instantiate the metamodel of our DSL as follow:
\begin{itemize}
\item Define a \textit{StructureFactory} in which the \textit{instances} property is a list which contains the objects \textit{list0} and \textit{list1}.
\item Instantiate a \textit{StructureType} where the built-in variable \textit{initialObjects} receives as value a list with one member - list0 or list1.
      The other properties of this \textit{StructureType} instance are detailed in the next steps: 
      \begin{enumerate}
      \item Define a variable \textit{n} with initial value zero.
      \item Define a membership function which return true if an object is instance of \textit{NodeEntry} and it is referenced by an object for which the membership function also return true. Observe how this recursive function returns true for each element in \textit{list0} because the initial object's list contains object \textit{list0}. 
      \item Update the variable \textit{n} by increasing its value by one.
      \end{enumerate}  
\end{itemize}
Further details about this example are discussed in the next section.

\subsection{Concrete Syntax}\label{sec:concrete-syntax}

%\setlength{\grammarparsep}{20pt plus 1pt minus 1pt}
{
\scriptsize
\begin{figure}[!ht]
\begin{mdframed}[outermargin=0.5cm, innermargin=0.5cm]

\newcommand{\grule}[1]{\hfill{\scriptsize (#1)}}
\setlength{\grammarindent}{5em}
\begin{grammar}

<program> ::= <types> <structures> \grule{1}

<types> ::= <type> <types> | <empty> \grule{2, 3}

<type> ::= <id> `:' `table-of' <id> | <id> `:' `struct' `{' <fields> `}' \grule{4, 5}

<fields> ::= <id> `:' <id> <fields> | <id> `:' <id> \grule{6, 7}

<structures> ::= <struct-factory> <structures> | <struct-factory> \grule{8, 9}

<struct-factory> ::= `foreach' <id>`:'<expr> `create' <body> \grule{10}

%<Header> ::= `create structure foreach' <id>`:'<expr>

<body> ::= <constructor> `membership' <expr> <updates> \grule{11}

<constructor> ::= `constructor' <statements> \grule{12}

%<membership> ::= 

<updates> ::= `updates' <statements> \grule{13}

<statements> ::= <s> <statements> | <s> \grule{14}

<s> ::= <id> `=' <e>  \grule{16} 

<e> ::= <e> <binary-op> <e> | <unary-op> <e> \grule{17a, 17b}
\alt <e> `in' <id> | <e> `is' <id> \grule{17c, 17d}
\alt <e> `.' <id> `(' <expr-list> `)' \grule{18}
\alt <e> `.' <id> `(' `[' <id> `|' <statements> `]' `)' \grule{19}
\alt <e> `.' <id> \grule{20}
\alt `#[' <expr-list> `]' | `struct' <id> `{' <expr-list> `}' \grule{21, 22}
\alt <int-literal> | <string-literal> | <bool-literal> \grule{23, 24, 25}
\alt <id> | `(' <e> `)' \grule{26, 27}

<expr-list> ::= <e> `,' <expr-list> | <empty> 

<binary-op> ::= `+' | `*' | `-' | `/' | `and' | `or'

<unary-op> ::= `-' | `not'


\end{grammar}
\end{mdframed}
\end{figure}
}

\mathlig{->}{\rightarrow}
\mathlig{|-}{\vdash}
\mathlig{=>}{\Rightarrow}
\mathlig{"}{\rho}

\mathligson

% classics
\[
% true
\inference[It(25):]{}{so, E, S|-true:Bool(true),S}
\quad\quad
% false
\inference[It(25):]{}{so, E, S|-false:Bool(false),S}
\]

\[
% integers
\inference[It(23):]{\textit{i is an integer literal}}{so, E, S|-i:Int(i),S}
\quad\quad
% strings
\inference[It(24):]{\textit{s is a string literal of length l}}{so, E, S|-s:String(s,l),S}
\]

\[
% id
\inference[It(26):]{
l_{id}=E(id) \\
v = S(l_{id})
}{so, E, S|-id:v,S}
\quad\quad
% strings
\inference[It(27):]{
so, E, S|-e:v,S_1
}
{so, E, S|-(e):v,S_1}
\]

\[
% unary operators
\inference[It(17b):]{
so, E, S|-e_0:v_0,S_1
}{so, E, S|--e_0:-v_0,S_1}
\quad\quad
% binary operators
\inference[It(17a):]{
so, E, S|-e_0:v_0,S_1 \\
so, E, S_1|-e_1:v_1,S_2
}{so, E, S|-e_0+e_1:v_0+v_1,S_2}
\]

\[
% in operator
\inference[It(17c mal):]{
so, E, S|-e_0:v_0,S_1 \\
so, E, S_1|-e_1:v_1,S_2
}{so, E, S|-e_0 \; in \; id:v,S_2}
\quad\quad
% is operators
\inference[It(17d):]{
so, E, S|-e_0:v_0,S_1 \\
v_0 = X(a_1=l_1, \ldots, a_n=l_n) \\
v = \textit{X is subtype of T?} \; Bool(true) \textit{:} Bool(false) \\
}{so, E, S|-e_0\; is \; T:v,S_1}
\]

\[
% assignment
\inference[It(16):]
{
so,E,S|-e_0:v_0,S_1\\
l_{id} = E(id) \\
S_2 = S_1[v_0/l_{id}]
}
{so, E, S|-id = e_0:v_0, S_2}
\quad\quad
% statements
\inference[It(14)]
{
so, E, S|-s_1:v_1,S_1 \\
so, E, S_1|-s_2:v_2,S_2 \\
\dots \\
so, E, S_{n-1}|-s_n:v_n,Sn
}
{so, E, S|-s_1,\dots,s_n:v_n,S_n}
\]

\[
% struct literal
\inference[It(22):]
{
so,E,S|-e_1:v_1,S_1 \\
\ldots \\
so,E,S_{n-1}|-e_n:v_n,S_n \\
class(T)=(a_1:T_1, \ldots, a_n:T_n) \\  % take the fields of the object
l_i = newloc(S_n) \; for \; i = 1 \ldots n \\
v=T(a_1=l_1, \ldots, a_n=l_n) \\ % assign locations to fields
S_f = S_n[v_1/l_1, \ldots, v_n/l_n]
}
{so, E, S|-struct\;T\;\{ e_1, \ldots, e_n \}:v,S_f}
\quad\quad
% list literal
\inference[It(21):]
{
so,E,S|-e_1:v_1,S_1 \\
\ldots \\
so,E,S_{n-1}|-e_n:v_n,S_n \\
l_i = newloc(S_n) \; for \; i = 1 \ldots n \\
v=Table(a_1=l_1, \ldots, a_n=l_n) \\
S_f = S_n[v_1/l_1, \ldots, v_n/l_n]
}
{so, E, S|-\#[e_1, \ldots, e_n]:v,S_f}
\]

\[
% accessing field
\inference[It(20):]{
so,E,S|-e:v0,S_f \\
v0 = X(a_1=l_1, \dots, a_n=l_n) \\
l_{id} = l_i \; where \; a_i = id \\
v = S_1(l_{id}) \\
}
{so,E,S|-e.id:v,S_f}
\quad\quad
% calling method
\inference[It(18):]{
so,E,S|-e_1:v_1,S_1 \\
\ldots \\
so,E,S_{n-1}|-e_n:v_n,S_n \\
so,E,S_n|-e_0:v_0,S_{n+1} \\
v_0 = X(a_1=l_1, \dots, a_n=l_m) \\
impl(X,f) = (x_1, \dots, x_n, f_{body}) \\
l_{x_i} = newloc(S_{n+1}) \; for \; i = 1 \dots n \\
E' = [a_1:l_1, \dots, a_m:l_n][l_{x_1}/x_1, \dots, l_{x_n}/x_n] \\
S_{n+2} = S_{n+1}[v_1/l_{x_1}, \dots, v_n/l_{x_n}] \\
v_0, E', S_{n+2}|-f_{body}:v,S_f
}
{so,E,S|-e_0.f(e_1, \ldots, e_n):v,S_f}
\]

% news

% lambda expressions
\inference[It(19) mal]{so,E,S|-statements:v,S_1}{so,E,S|-[id|statements]:v,S_1}

A textual concrete syntax has been defined for our DSL allowing the domain expert to define a custom memory profiler.
As an example, the next listing shows how to compute the length of each \textit{SimplyLinkedList} in the memory snapshot depicted in Figure~\ref{fig:simple_snapshot}.
The mechanism is based on counting the number of \textit{NodeEntry} referenced by the \textit{SimplyLinkedList}.
Each \textit{NodeEntry} is used to wrap one element of data and to point to the next element.

\begin{lstlisting}[escapeinside={(*}{*)}, 
label=lst:listLengt, language=DSL2
]
create structure foreach e:objects.filter(l| l is SimplyLinkedList) using
  constructor
    initialObjects = #[e] // a list literal with one element: e
    n = 0
  membership (this is NodeEntry) and (referrer in this_structure)
  updates 
    n = n + 1
\end{lstlisting}

In the example, only one \textit{StructureFactory} is necessary.
In line 1, we define the list of structures we are interested in.
We do so by selecting instances of class \textit{SimplyLinkedList} as elements of the \textit{StructureFactory}.
Since there are two simply linked lists in the memory snapshot, we are going to build two structures.
Observe the usage of a built-in \textit{rvalue} named \textit{objects} which contains all the objects in memory.
The valid scope of this rvalue is in both the definition of the set of structures and the computation of the initial values.
Thereafter, lines 2-4 specify the initial values.
Line 3 in particular initializes the set of objects included in the structure.
Notice the usage of a list literal to include the object referenced by \textit{e}.
In the initialization scope, the rvalue \textit{e} is equal to one of the element within the list of structures - either \textit{list0} or \textit{list1}.

In line 5 we define the membership function, which is used to determine if an object is part of the structure.
There are four built-in rvalues during the evaluation of the function as well as during the update of the variables.
First, the value named \textit{this} is the current visited object. 
The \textit{membership} boolean \textit{expression} aims at determining if this object is part of the \textit{structure} or not. 
The value \textit{this\_structure} identifies the structure.
As our runtime profiler will traverse the graph of in memory objects following references between objects, the object through which we have reached the \textit{this} object is known as \textit{referrer}.
The last value, which is target dependent, is the kind of reference.
Operator \textbf{is} checks if \textit{this} is an instance of class \textit{NodeEntry}.
Likewise, operator \textbf{in} checks whether the \textit{referrer} is already a member of the structure.
Finally, line 7 updates the length of the list when an object is detected as member of the structure.

\subsection{Translational Semantics}\label{sec:semantic}

An instance of our metamodel is compiled into a custom memory profiler.
This compilation produces a library written in \textit{C++} which is in charge of collecting the desired information from the runtime environment.
The generated source code has two parts.
First, for each \textit{StructureType} in the model, the compiler generates a subclass of \textit{AbstractStructureType} which is shown below.
Every subclass contains attributes to store the variables used in the associated \textit{StructureType}.
In the listing below, the class \textit{Context} holds the built-in values we mention in the previous section.
\begin{lstlisting}[language=C++, frame=tblr,
numbers=left,
numberstyle=\color{black}\scriptsize,]
class AbstractStructureType {
public:
	void initialize(Context& ctx) = 0;
	bool membership(Context& ctx) = 0;
	void update(Context& ctx) = 0;
}
\end{lstlisting}

The second part of the generated code is formed by a set of initialization routines, one for each \textit{StructureFactory}.
Each routine creates a list of structures with a specific \textit{AbstractStructureType} - the \textit{T} parameter in the listing.
Formally, the signature and behavior of these routines are as follow:
\begin{lstlisting}[language=C++, frame=tblr,
numbers=left,
numberstyle=\color{black}\scriptsize,]
template <typename T> void
[name](Context& ctx, std::vector<AbstractStructureType*>& s){
  for (Object obj : ctx.instances) {
    AbstractStructureType* ns = new T();
    ctx.e = obj;
    ns->initialize(ctx);
    s.add(ns); // not valid in the STL, but simpler to read
  }
}
\end{lstlisting}
An important concern during the transformation lies on efficiently mapping our concepts to \textit{C++} concepts.
Moreover, since each target platform provides facilities to get metadata regarding the objects in memory, using such facilities efficiently is specially important in order to reduce the performance overhead due to profiling.

The final profiler is built using both the generated code and a template algorithm.
The template is target dependent, but in general we use the underline target facilities to collect meta-data, access fields in certain steps, traverse the objects in memory and also to populate the built-in rvalues.
A simplified version of the used algorithm is shown below:
\begin{lstlisting}[escapeinside={(*}{*)},
frame=tb, label=lst:template, language=AlgLang,
numbers=left,
numberstyle=\color{black}\scriptsize]
values:
   structures: vector<AbstractStructureType*>
routine:
   foreach (initialization rountine (*$R_i$*) associated to a StructureFactory)
      create context
	  call (*$R_i$*)(context, structures)
   foreach (r: references among objects)
      if (r.target has no membership)
         create context // context.this = r.target
         S = structures.findfirst(s | s.membership(context))
         make context.this a member of S
         S.update(context)
   return structures 
\end{lstlisting}
There are two loops in the algorithm. 
The former loop is in charge of creating the set of structures the program is intended to collect information about.
The creation of the context in line 5 depends on the target platform.
It basically creates values such as the list of objects in memory or the list of loaded classes.
The latter loop traverses all the references among objects in memory.
During each iteration, the algorithm finds the first structure for which the membership function is true.
Notice that we only select the first because for some memory accounting problems, it is too hard to define a membership functions that build disjoints structures~\cite{dsn/09/geoffray/ijvm,Attouchi:2014:MMM:2602458.2602467}.
Thereafter, the information for such a structure is updated.

%The complete execution of a program in our language is as follow. Subgraph instances are initialized using listing~\ref{onInitialization}.
%Afterward, all the references in the graph of objects are traversed running listing~\ref{lst:onNodeFoundData}.
%The output data for all subgraph instances has been collected after all the references are traversed once.

%Finally, in listings~\ref{lst:onNodeFound},~\ref{lst:onNodeFoundData} and~\ref{onInitialization} we use built-in properties that are defined by the user.
%These properties must have access to some data describing the content of the memory in order to successfully identify subgraphs and calculate output values.
%Such data is wrapped in what we call execution context.
%In our DSL there are two different execution contexts: \textit{global context} and \textit{local context}.
%The former includes built-in values such as: i) lists of \textit{objects}, \textit{threads}, \textit{classes}, etc. , and ii) a value called \textit{Entity} representing a subgraph instance.
%The latter only contains the object \textit{THIS} which is being visited, a label of the reference representing its type, \textit{REFERRER} which is the object referencing the visited and again the \textit{Entity} value.
%The \textit{global context} is available in listing~\ref{onInitialization} while the \textit{local context} is available in both listings~\ref{lst:onNodeFound} and~\ref{lst:onNodeFoundData}.

\subsection{Language Usage}

There are several possibilities for using our DSL in the various stage of an application lifecycle.
These include checking: local data structure invariants, reachability properties, memory consumption properties and combinations of those.
Below, we show some examples to highlight possible usage of our language. 
 
The first example shows how to assert the existence of a value satisfying some properties, independently of which object contains it. 
The result is obtained through the use of a filter on the list of objects.
The assertion successes if the heap contains an object with an attribute named $data$ with a value comprised between $3.141$ and $3.142$.

\begin{lstlisting}[escapeinside={(*}{*)},
%label=assertion, 
language=DSL2]
create structure foreach e:#["whole-jvm"] using
  constructor
    initialObjects = #[]
    existValue = false
  membership true
  updates 
    existValue = existValue or (this.data > 3.141 and this.data < 3.142)
\end{lstlisting}

The goal of next listing is to detect a bug identified in~\cite{Aftandilian:2009:GAU:1543135.1542503}.
This listing aims at finding if there exists an instance of the class $Order$
with the value of its field $field$ being equal to $specialValue$.
This technique is used to detect if one object has been garbage collected or if someone still holds a reference on it preventing its garbage collection.

\begin{lstlisting}[escapeinside={(*}{*)},
%caption=Detecting a knwon bug in pseudojbb., 
%label=pseudojbb,
%float=!h,
language=DSL2]
create structure foreach e:#["whole-jvm"] using
  constructor
	initialObjects = #[]
	fault = false
  membership  true
  updates
    fault = fault or (this is Order and this.field = specialValue)
\end{lstlisting}

The next example computes a combination of reachability and memory consumption properties.
It calculates the number of objects, and their total memory consumption, that are reachable from the threads.
We can notice how the membership property discards those objects that are not referenced by an already included object.  

\begin{lstlisting}[escapeinside={(*}{*)},
caption={Calculating objects reachables from threads},
label=kevoreeaccounting,
%float=!h,
language=DSL2]
create structure foreach e:#["whole-jvm"] using
  constructor
    initialObjects = threads
    nbObjects = 0 
    nbSize = 0 
  membership ( this in None and referrer in this_structure )
  updates 
    nbObjects = nbObjects + 1  
    nbSize = nbSize + this.size
\end{lstlisting}

We can also express complex structures in memory.
For instance, to find the consumption of K3-Al object namely \textit{K3Object} as described in Section~\ref{sec:chapter2-introduction}, we must find all instances of \textit{HashMap.Entry} that have \textit{K3Object} as the \textit{key}. These entries should be added to the consumption of the object \textit{K3Object} as well as all the objects reachable from the \textit{HashMap.Entry.value}.
To come out with this solution a good understanding of how K3-Al implements aspects is required.
The rationale here is that K3-Al stores the state of aspects in a separate HashMap, using as key the object to be aspectized.

\begin{lstlisting}[escapeinside={(*}{*)},
%caption=Computing the consumption of each K3-Al Object along with its aspects., 
%label=k3,
%float=!h, 
language=DSL2]
create structure foreach e:objects(*{.filter}*)([it|it is K3Object]) using
  constructor
    initialObjects = #[e]
    nbSize = 0
  membership (referrer in this_structure and this in None ) or
    (this is HashMap.Entry and this.key in this_structure and 
      this.key is K3Object)
  updates
    nbSize = nbSize + 1
\end{lstlisting}

\section{Tooling}\label{sec:dsl-implementation}

To validate our approach, we have implemented a tool chain to ease the definition of custom memory profilers for Java-based systems.~\footnote{Available at: \url{https://github.com/intigonzalez/heapexplorer\_language}}
These profilers can be executed in any JVM as long as it provides support for the JVMTI.

In this section, we present tools built to support the definition of memory profilers using our language; this is done by taking into account how engineers in different roles may interact with these tools and with the resultant profilers.
Indeed, in dealing with memory profilers, we have to take into consideration the two usual roles -- developers of profilers and their users; after all, profilers built using our language are themselves software abstractions.
A developer must know how the target domain-specific abstractions are represented on top of the JVM, and she/he must also have a clear understanding of how our language is executed.
On the contrary, users only need to be aware of the interface provided by our framework, and the structure of the data collected by a profiler.
In the rest of this section, we discuss details that are important to these roles.

Additionally, we present low-level details regarding how the language is implemented atop of the JMVTI.
The decision of implementing our approach by relying on JVMTI has advantages and disadvantages.
On the one hand, the obvious advantage lies on the portability of this solution, which makes it more valuable from a practical point of view.
On the other hand, building profilers on top of the JVMTI, instead of directly modifying the JVM, impacts the performance of the generated profilers and, unfortunately, hinders (in extreme case it even prevents) the implementation of some language constructs.
Nonetheless, it is our belief that guarantying profilers' portability should be of maximum priority.
Moreover, in writing this implementation, we have found that the limitations in the JVMTI preventing the construction of better profilers can be overcome with, at most, a few additions to the API.


\subsection{Developers of domain-specific abstractions}

%\extracomment{FIX}{I am fixing this section, you can continue in Section~\ref{sec:dsl-tooling-users}}

In our vision, developers of software libraries and component frameworks, as well as software language engineers may use our approach to  define customized memory profilers for the abstractions they create.
This is, in addition to delivering artifacts such as libraries, source code, simulators, text editors for DSLs, and compilers for these DSLs; engineers would also ship profilers to simplify the use of these abstractions.
For instance, the developers of the Spring framework~\footnote{\url{https://spring.io/}} may create a set of specific profilers to reduce the cost of maintaining applications written using the framework.
These profilers can serve as both internal tools to help in the development of abstractions, and mechanisms allowing users to better use abstractions.

Figure~\ref{fig:dsl-tooling-developer} summarizes the viewpoint of developers of domain-specific abstractions.
To write a profiler, they use knowledge about the abstraction and the tool chain to generate the executable profiler. 
Our implementation of the language is built using Xtext~\cite{Eysholdt:2010:XIY:1869542.1869625}; it provides a textual editor that is able to handle the proposed concrete syntax.
This editor provides syntax highlighting, error detection during editing, auto-completion, and compilation to native Java agents written in \textit{C++}.

\begin{figure}
\centering
\includegraphics[scale=0.45]{./chapter6/fig/developer-profiler-view.png}
\caption{Developer viewpoint. Memory profilers are built from the description of software abstractions.}\label{fig:dsl-tooling-developer}
\end{figure}

To perform low-level tasks related to memory profiling, we use \glslink{JVMTI}{JVMTI}~\footnote{\url{http://docs.oracle.com/javase/8/docs/platform/jvmti/jvmti.html}} and \glslink{JNI}{JNI}.
These APIs are used by both profilers and the core memory profiling library, so-called Native Agent in Figure~\ref{fig:dsl-tooling-developer}.
In this native agent, a \textit{plugins} system, which allows users to load/unload profiles without shutting down the JVM, is implemented.
Given a profiler definion, the compiler output is a package that contains a package with the native binary code for the profiler, and a Java library you can use to access the collected data using plain Java objects.

To reduce the overhead of profilers, developers must be aware of the details of the abstraction for which the profiler is being built, the semantic of our language, and the details of its implementation.
In particular, it is advisable reducing the usage of \textit{lists} and the evaluation of nested lambda expressions.
Likewise, heavily using the built-in rvalue \textit{objects} is specially discouraged because it can easily contain many elements.
It is also discouraged because, in order to reduce memory consumption, we rely on an iterator built on top of JVMTI operations that can be costly to use in terms of CPU time.

Finally, we added some built-in rvalues in this implementation because they are both useful in the context of Java and easy to obtain using the JVMTI.
These values are \textit{classes}, \textit{classloaders}, \textit{threads} and \textit{objects}; they are lists of anonymous built-in record types.
The relations among these types and their operations are depicted in Figure~\ref{fig:dsl-built-in-types}.

\begin{figure}
\centering
\includegraphics[scale=0.45]{./chapter6/fig/diagram-classes.png}
\caption{Viewpoint of developers. Memory profilers are built from the description of software abstractions.}\label{fig:dsl-built-in-types}
\end{figure}

%Indeed, the process of code generation is driven by the need of reducing the performance impact.
%In our implementation, we apply a set of platform dependent optimizations taking into account the profiler description.
%First, since a profiler does not always need built-in rvalues (e.g., \textit{threads}, \textit{threadgroups} and \textit{classes}, etc.), we selectively skip the construction of them.
%When possible, we also skip the construction of some structures (e.g., class of each object, its classloader, field names, etc.).

\subsection{Users of domain-specific abstractions} \label{sec:dsl-tooling-users}

We envision that a set of memory profilers can be shipped in addition to other ``classic'' deployment artifacts that users of a software abstraction receive. 
These profilers would support the use of the corresponding software abstraction.
For instance, a user who is relying on a new extension of the Xtend language to build a system, might use specific profilers written in our language to understand the memory consumption, and in general, the behavior of the system.

The generated profilers can be used in two different ways, either as development tools or as mechanisms to support resource awareness at runtime.
Due to the scope of this thesis, the reference implementation we provide is biased towards the second scenario, but it should be relatively simple to adapt it to support the software development process.
To access memory profilers, a JVM must be launched with a native Java agent loaded, and a library to collect profiling data in its classpath.
Once the application is running, it can trigger profiling by simple issuing a few method calls using the profiling API.
Figure~\ref{fig:user-profiling-library-view} illustrates the process of collecting memory profiles, the software components involved, and the APIs that must be used.
Observe how the profiling framework issues a call to a handler once it is done, a parameter contains the data computed.
These data are encoded in a \textit{list}, in which each elements correspond to the data computed for each identified structure in the heap.
%The problem is knowing the type and shape of each list element.

\begin{figure}[!b]
\centering
\includegraphics[scale=0.6]{./chapter6/fig/user-profiler-view.png}
\caption{Viewpoint of users. Memory profilers are black-boxes accessed through Java interfaces. Data collected is in the form of plain Java objects.}\label{fig:user-profiling-library-view}
\end{figure}

The output of a profiler is a list of Java objects containing the collected information; and the type of these objects depend on the profiler definition.
Indeed, as part of our implementation, the profiler generator creates a set of Java classes to represent the data collected in a form that is easy to digest at runtime by a Java application.
Once a profiler collects the information in an internal format, it populates a representation in Java using the Java Native Interface (JNI); the code to do so is also generated by the compiler of our language.
In Figure~\ref{fig:dsl-generated-java}, the classes generated for a profiler are shown.
Notice that a class is created for each \textit{record} declared, and also for each \textit{StructureType}.
It can be seen how `lists' are directly represented in Java by mean of generic Java lists.
The \textit{id} field in both \textit{MemoryProfile1} and \textit{MemoryProfile2} is the value used to parametrize each structure;
in this particular example, where two structures are identified, the value of \textit{MemoryProfile1.id} is ``lists'' and the value of \textit{MemoryProfile2.id} is ``otherObjects''.

Given the fact that the data computed by a profiler is returned as a list of objects, and their layout is unclear, the remaining problem is how to process such data; there are two options.
First, users can make the application code depend on the Java code created by the profiler generator.
In this way, your application has a new dependency, but you can profit from knowing at development time the types used in the code.
A second approach is using the reflection capabilities of Java to explore the data.
In the evaluation, we use such an approach to log the result of an arbitrary profiler, printing all the information it has computed.
Using reflection, it is also possible to build a user interface to explore the results in a customized way.  

\begin{figure}
\centering
\begin{minipage}[t]{0.60\linewidth}
\begin{lstlisting}[language=DSL2]
name "basic info" 

T : struct {
	classname : String
	size: int
}
create structeres for e:#["lists"]
using
	constructor
		initialObjects = #Object[]
		data1 = #T[];
	membership
		(this is String) or (this is Array)
	updates
		data1 = data.add(struct T { this.classname, this.size})
		
create structeres for e:#["otherObjects"]
using
	constructor
		initialObjects = #Object[]
		data2 = #T[];
	membership true
	updates
		data2 = data.add(struct T { this.classname, this.size})
\end{lstlisting}
\end{minipage}
\hspace{0.07\linewidth}
\begin{minipage}[t]{0.30\linewidth}
\begin{lstlisting}[language=java, frame=L, numbers=left,numberstyle=\color{black}\scriptsize]
class T {
	final String classname;
	final int size;
}

class MemoryProfile1 {
	final Object id;
	final List<T> data1;
}

class MemoryProfile2 {
	final Object id;
	final List<T> data2;
}
\end{lstlisting}
\end{minipage}
\caption{Representation of profiling data in Java, as users of profilers see it. Accessing these structures is useful to support resource awareness.} \label{fig:dsl-generated-java}
\end{figure}




%The process of code generation is driven by the need of reducing the performance impact.
%In general, there are two ways of optimizing the impact of the memory's analysis.
%First, we can apply platform dependent optimizations.
%The second option is to apply platform independent optimizations; for instance, simplifying the evaluation of each expression.
%In our implementation, we use both platform dependent and independent optimizations.
%
%A set of platform dependent optimizations we perform is related to the construction of the built-in values \textit{threads}, \textit{threadgroups}, \textit{classes}, etc. 
%Since not all memory analysis depends on such values, we selectively skip the construction of them.
%For instance, in listing~\ref{assertion} there is no need to compute any of such values as is also unnecessary to identified the class of each object.
%Extending this idea to other cases (e.g., class of each object, its classloader, field names, etc.) is straightforward.
%To implement these optimizations, we used a parametrized code template, so the code generate depends on the values of these parameters which we can tune to satisfy our needs.
%
%An other optimization we perform is related to the existence of collections as data type in our DSL.
%These collections can be potentially large, in particular, the \textit{objects} value is costly to compute and keep in memory.
%This fact combined with the usage of operations on collections such as \textit{map} and \textit{filter} may harm the performance of an analysis.
%That is why we devise two strategies to deal with collection values.
%User-defined and most built-in collections are kept in memory using linear space.
%On the contrary, we represent the built-in collection \textit{objects} as a generator.
%This representation is feasible because the mechanism provided by JVMTI to access the objects is based on callbacks.
%
%A last optimization is reducing the nodes of the graph that must be traversed
%As an illustration, we only produce code to explore primitive fields of each object, which are represented as leaf nodes in the graph, if there exist some expression accessing a field.
%
%As for platform independent optimizations, we mostly change the order in which boolean expressions are evaluated.
%We try to guarantee that subexpressions accessing collections and fields are evaluated as little as possible.
%
%The current implementation is limited in the number of optimization it applies.
%The main overhead reduction is achieved thanks to the execution model in which many paths of the graph are not traversed.
%Other benefits come from deciding at compilation time if some parts of the graph such as the leaf nodes must be explored or not.


\section{Discussion On Language Expressiveness}\label{sec:expressiveness}

A main feature of our approach is that it makes explicit how the data is collected.
In short, our language follows the imperative paradigm when it comes to data collection.
However, many well-known and established languages to recover data provide a declarative style because it simplifies the coding of queries.

Designing a domain specific language for memory analysis is a deliberate attempt to make explicit to the user what is the complexity of the analysis he tries to perform.
We acknowledge that our approach limits,  for two reasons, the kind of memory analysis that users can express.
First, it is not possible to recover all the information contained in the graph of live objects in linear time on the set of objects.
Second, an imperative style forces the users to understand the underlying execution model which is not required with declarative query languages.
Nonetheless, we claim that getting rid of some expressiveness is a trade-off worth to consider in order to guarantee efficient memory analysis.
The rational behind this assumption is that of the two strategies we followed to achieve efficient memory analysis, namely: i) generating efficient native code to collect the data, and ii) reducing the data recollection capabilities by carefully designing the language; the second one has a bigger impact on the performance gain.

At this point, it is worthy noting why declarative approaches fail to deliver the adequate performance in production.
Listings~\ref{k3OQL} and~\ref{k3Cypher} show possible solutions, in OQL and Cypher/Neo4j, to the K3-Al example presented in section~\ref{sec:motivation}.
There are two aspects affecting the performance of this kind of queries.
We next discuss them.

\begin{lstlisting}[escapeinside={(*}{*)},caption={Using OQL to compute the consumption of each K3-Al Object along with its aspects. Actually, this query cannot be executed in Eclipse Mat nor in VisualVM since they do not provide a complete OQL implementation. }, label=k3OQL,float=!h, language=OQL]
SELECT id, sum(size) as s
FROM (
	SELECT
		e.key.@objectId AS id, 
		e.@usedHeapSize + e.value.@retainedHeapSize AS size
	FROM java.util.HashMap$Entry e
	WHERE (classof(e.key).@name = "K3Object")
	UNION ALL
	SELECT 
		k3.@objectId AS id, k3.@retainedHeapSize AS size 
	FROM K3Object k3
)
GROUP BY id
\end{lstlisting}

First, many queries are intrinsically complex to answer.
For instance, it is known that answering SPARQL queries - which was used as inspiration for Cypher/Neo4j, is PSPACE-complete~\cite{Schmidt:2010:FSQ:1804669.1804675, Perez:2009:SCS:1567274.1567278}. 

The second aspect affecting the performance of declarative queries for in production memory analysis lies in the nature of data we are exploring. 
Indeed, even if declarative queries can be executed efficiently, the optimization steps required are in most cases impossible to execute for the type of data we are considering - a graph of objects that constantly changes.
Often, these query optimizations require access to indexes, additional storage and multiples passes on the data~\cite{Elhemali:2007:ESS:1247480.1247598, Dageville:2002:SMM:1287369.1287454} that are not accessible on the graph of objects.

%\todo{cite: Foundations of SPARQL query optimization}

\begin{lstlisting}[escapeinside={(*}{*)},caption=Using Cypher/Neo4j to compute the consumption of each Kermeta 3's Object along with its aspects., label=k3Cypher,float=!h, language=CYPHER]
MATCH 
	(key:K3Object)<-[:key]-(entry:HashMap$Entry)-[:value]->value
WITH entry, key, value
MATCH 
	key-[:1..1]->fieldK
WITH entry, key, value, fieldK
MATCH 
	value-[:1..1]->fieldV
RETURN key, entry.size + key.size + fieldK.size + sum(value.size) + sum(fieldV.size);
\end{lstlisting}

On the contrary, as we already mention, our language makes explicit both the time and space complexities of the analysis.
We also believe that the mental model required to code memory analysis with our DSL is simple enough since
we have tried to mimic the ``think as a vertex'' paradigm of Pregel which has proven to be successful~\cite{Malewicz:2010:PSL:1807167.1807184}.

%\begin{lstlisting}[escapeinside={(*}{*)},caption=Detecting a knwon bug in pseudojbb., label=pseudojbbOQL,float=!h, language=OQL]
%SELECT o
%FROM Order o
%WHERE o.field = specialValue
%\end{lstlisting}

\section{Evaluating performance of profilers}\label{sec:dsl-evaluation}
%\todo{Add research questions}

In this section, we evaluate the implementation of our approach.
To do so, we present experiments to measure the performance overhead induced by memory profilers built using the proposed approach.
This section aims at assessing whether our approach induces low overhead across different applications and types of analysis.
Indeed, using memory profilers that have different levels of complexity, makes our evaluation closer to the expected use in real-world scenarios.

The goal of this section is answering the following research questions:
\begin{enumerate}
\item \textbf{RQ1. Does our approach produce profilers with lower overhead than state-of-the-art tools when used to perform many iterations of memory analysis at runtime?} To answer this question, we assess the overhead on total execution time produced by the periodic computation of a specific analysis.
In this experiment, we measure and compare the overhead of our approach against the overhead produced by other solutions.
\item \textbf{RQ2. Is significant the difference between the time needed to execute a single analysis with our approach in comparison to previous solutions? }
In a second experiment, we measure the execution time needed to perform a single memory analysis step instead of focusing on the total application execution time.
\item \textbf{RQ3. Does the advantage of our approach remain for real applications? }
 Finally, we perform memory analysis on actual applications, including \textit{Eclipse}, \textit{NetBean}, and others, to assess the overhead of profiling in ``real-life'' scenarios. 
\end{enumerate}

In general, these experiments show that our language produces specific profilers with lower overhead for applications running in production environments than well-known memory profilers.

\subsection{Methodology and Setup}\label{sec:MethodologyAndSetup}
Our system is implemented on top of the JVMTI; thus, we compute our results using the HotSpot JVM version 1.7.0\_76, with a heap size of 2GiB for all the experiments.
Across this section, we use Eclipse Memory Analyzer 1.4.0 (Eclipse MAT), a production ready memory profiler, to perform several experiments.
We use this tool in \gls{CLI} mode; this executes the desired analysis in a separate process.
In other words, in performing a memory analysis on a JVM instance \textit{A}, we dump its heap and invoke Eclipse MAT in a separate JVM instance to collect profiling data.

We use DaCapo benchmarks version 2006-10-MR2~\cite{Blackburn:2006:DBJ:1167473.1167488} in the first two experiments, large input size in the first experiment, and default input sizes in the second one.
In the third experiment, we use a set of actual applications based on OSGi, these applications are listed in the relevant section~\footnote{Links are available at \url{https://en.wikipedia.org/wiki/OSGi}}.
Although the details are specific to each experiment, in general, each measurement presented is the average of several runs under the same conditions.

To obtain comparable and reproducible results, we used the same hardware across all experiments: a 2.90GHz Intel(R) i7-3520M processor, running Linux with a 64 bit kernel version 3.17.3 and 8GiB of system memory.

\subsection{Impact of Analysis on the Total Execution Time}

In this experiment, we assess how much our approach affects the execution time of applications.
To do so, we compare the time reported by the execution of DaCapo benchmarks without any kind of memory analysis against the execution time when our language is used to perform the analysis in Listing~\ref{lst:kevoreeaccounting}.
In addition, we check how our approach behaves in comparison to other approaches for memory analysis.
In this case, the profiler finds the number of objects, and their total size, when threads are used as only roots to traverse the graph of live objects.

The experiment was configured as follows: within a JVM instance, we wrap the execution of the DaCapo Benchmark.
Each DaCapo test is configured to execute 20 warm-up iterations before the final test execution.
This number of warm-ups is used to guarantee a long enough execution time.
A separate thread periodically performs a \textit{memory consumption monitoring step} every 2 seconds by using one of the methods we want to compare: 

\begin{description}
\item[No analysis] In this case, we simply execute the DaCapo Benchmarks without any additional task affecting its performance. This is the baseline for the comparison.  

\item[Handwritten JVMTI] In this solution, we traverse all references in the graph of live objects starting on the threads, during this process the JVM is fully halted, impacting the total application's execution time.

\item[Our approach] We use our language to define the profiler in Listing~\ref{lst:kevoreeaccounting}. It is compiled and used at runtime.

\item[Heap Dump + Eclipse MAT] This method uses the approach described in Section \ref{sec:MethodologyAndSetup}; when an analysis is required, the JVM dumps the heap and executes Eclipse MAT in a separate process in CLI mode.
\end{description}

%We then record the total execution time; in other words, the time required for the 20 warm-ups plus the time used in the final test.
%The idea is to check how much the performance is affected by each method.


\begin{figure}[!ht]
\centering
\begin{tikzpicture}
\begin{axis}[ybar=0pt, legend style={at={(0.72,1)},
every axis legend/.append style={nodes={right}},
anchor=north,legend columns=1, font=\tiny},
ylabel={Overhead (\%)},
y label style={at={(0.02, 0.5)}},
scaled y ticks = false,
      y tick label style={/pgf/number format/fixed,
      /pgf/number format/1000 sep = \thinspace % Optional if you want to replace comma as the 1000 separator 
      },
xtick=data,ymin=0,
width = 0.9\columnwidth,
height = 4.2cm,
bar width = 7,
x tick label style={rotate=45,anchor=east},
 axis lines*=left, % Don't display the top and right lines
 symbolic x coords={antlr,fop,hsqldb,jython,chart,luindex,xalan,lusearch, pmd, eclipse}
]
\addplot coordinates 
	{(antlr,3.9781514264) (fop,4.605707750) (hsqldb,29.2388250106) (jython,1.3401924419) (chart,2.9126870659) (luindex,7.4126736676)
	(xalan,3.5175679043) (lusearch,2.1071653048) (pmd,2.1071653048) (eclipse,13.2922104461) };
\addplot coordinates 
	{(antlr,4.6792415918) (fop,10.920169369) (hsqldb,33.4078193658) (jython,7.3669103815) (chart,10.0961181121) (luindex,5.8949045922) 
	(xalan,10.6595492114) (lusearch,8.8185623499) (pmd,11.7847827707) (eclipse,15.7219232736)};
\addplot coordinates 
	{(antlr,28.7859273871) (fop,23.7271506764) (hsqldb,46.0448750552) (jython,32.4395399802) (chart,44.6349538836) (luindex,31.9874187461) 
	(xalan,37.7533619117) (lusearch,12.9664891096) (pmd,33.9112866499) (eclipse,32.6863711858)};
\legend{Handwritten JVMTI, Our approach, Heap Dump + Eclipse MAT}
\end{axis}
\end{tikzpicture}
\caption{Overhead on execution time compared to the execution without memory analysis for different tests in the DaCapo Benchmark\label{fig:evaluationTotalTime}}
\end{figure}

In this experiment, we measure the total time needed to complete the 20 warm-up iterations plus the time required to execute the final test.
The idea is to check how much the performance is affected by each method.
We repeat this process 10 times for each test in the DaCapo Benchmark suite, and take the average as final measurement.

It is useful to discuss how varies the number of times the analysis is performed.
As we mentioned, profilers runs periodically in this set of experiments; thus, the number of invocations to a profiler depends on the benchmark, and the overhead produced by the profiler itself.
For instance, using our approach, the memory analysis is executed a minimum of 10 times in the \textit{fop} benchmark, and a maximum of 366 times in the \textit{eclipse} benchmark.

Figure~\ref{fig:evaluationTotalTime} depicts the overhead in total execution time for different profiling strategies and Dacapo tests.
The values are shown as the percentage with respect to the baseline, which in this case is obtained when \textit{no analysis} is executed.
It is noteworthy that our approach performs close to the handwritten solution.
Moreover, our solution outperforms the \textit{Heap Dump + Eclipse MAT} approach even when the latter is executing mostly on a separate process without halting the JVM during profiling.
The overhead in our approach remains between 4-33\%, and it is 11.93\% in average.

\subsection{Comparing Analysis Time for an Assertion}

In the previous section, we show the performance overhead on total execution time for different profiling mechanisms.
However, these mechanisms are not executed under the same conditions.
For instance, as we mention in Section~\ref{sec:dsl-implementation}, our implementation suspends the execution of the application while it performs the analysis.
On the contrary, the \textit{Heap Dump + Eclipse MAT} approach only suspends the application while dumping the heap, but the analysis is done in a separate process; hence, it likely runs in parallel.
Therefore, in this experiment, we measure only the \textbf{analysis time}, which is the amount of elapsed time from the beginning of analysis to its end.
To perform these experiments, we use again the Dacapo benchmarks.
Since the analysis time depends on the number of objects visited during the computation, in this experiment, we assess the behavior of our approach using a memory profiler that must iterate over all objects to complete.
For the same reason, we repeat the analysis using different input size for each benchmark; this implies that a different number of objects is found in memory.

The \textbf{assertion} used in this experiment checks \textbf{whether an instance of a specific class exists in the heap}.
The following listing shows how to implement such an assertion using our language.
By defining the membership function as the \textit{true} constant, we guarantee that all objects are visited.

\begin{lstlisting}[escapeinside={(*}{*)},
caption={Detecting if there exists an instance of a specific class.}, 
label=lst:SimpleAssertion,
language=DSL2]
create structure foreach e:#["jvm"], using
   constructor
      initialObjects = #Object[]
      exists = false
   membership  true
   updates
      exists = exists or (this is UnusedClass)
\end{lstlisting}

The setting of the experiment is as follow.
The DaCapo benchmark suite is used with two different input sizes, default and large.
Before the final test, twenty warm-ups are executed in order to ensure long enough execution time.
A separate thread periodically checks the assertion and records the analysis time.
The average analysis time along the complete execution of a benchmark (i.e., xalan, fop, ...) is used as data point.
Ten of these data points are obtained through repetition of the previous step and used as final measurement for a pair of benchmark and analysis approach.
As in the previous experiment, we use a handwritten JVMTI agents and an Eclipse MAT extension to check the assertion with those tools.


\begin{figure*}[!ht]
 \centering
 \begin{minipage}[t]{0.45\linewidth}
 \centering
\begin{tikzpicture}
\begin{axis}[
ybar=0pt, 
legend style={at={(0.72,1)},
every axis legend/.append style={nodes={right}},
anchor=north,legend columns=1, font=\tiny},
ylabel={Analysis Time (sec)},
y label style={at={(0.1, 0.5)}},
scaled y ticks = false,
      y tick label style={/pgf/number format/fixed,
      /pgf/number format/1000 sep = \thinspace % Optional if you want to replace comma as the 1000 separator 
      },
xtick=data,ymin=0,
width = \columnwidth,
height = 4.2cm,
bar width = 4,
x tick label style={rotate=45,anchor=east, font=\small},
 axis lines*=left, % Don't display the top and right lines
 symbolic x coords={antlr,fop,hsqldb,jython,chart,luindex,xalan,lusearch, pmd, eclipse}
]
\addplot coordinates 
	{(antlr,1.9781514264) (fop,1.605707750) (hsqldb,2.2388250106) (jython,1.3401924419) (chart,2.9126870659) (luindex,1.4126736676)
	(xalan,1.5175679043) (lusearch,2.1071653048) (pmd,1.1071653048) (eclipse,3.2922104461) };
\addplot coordinates 
	{(antlr,2.2781514264) (fop,1.805707750) (hsqldb,2.5388250106) (jython,1.6401924419) (chart,3.2126870659) (luindex,1.6126736676)
		(xalan,1.7175679043) (lusearch,2.4071653048) (pmd,1.3071653048) (eclipse,3.5922104461) };
\addplot coordinates 
	{(antlr,2.9781514264) (fop,2.605707750) (hsqldb,3.2388250106) (jython,2.3401924419) (chart,3.9126870659) (luindex,2.4126736676)
		(xalan,2.5175679043) (lusearch,3.1071653048) (pmd,2.1071653048) (eclipse,4.2922104461) };
%\legend{Handwritten JVMTI, Our approach, Heap Dump + Eclipse MAT}
\end{axis}
\end{tikzpicture}
\caption{Analysis time with default input size\label{fig:analysisTimeDefaultSize}}
\end{minipage}
\hspace{0.05\linewidth}
\begin{minipage}[t]{0.45\linewidth}
 \centering
\begin{tikzpicture}
\begin{axis}[ybar=0pt, legend style={at={(0.28,1.13)},
every axis legend/.append style={nodes={right}},
anchor=north,legend columns=1, font=\tiny},
ylabel={Analysis Time (sec)},
y label style={at={(0.1, 0.5)}},
scaled y ticks = false,
      y tick label style={/pgf/number format/fixed,
      /pgf/number format/1000 sep = \thinspace % Optional if you want to replace comma as the 1000 separator 
      },
xtick=data,ymin=0,
width = \columnwidth,
height = 4.2cm,
bar width = 4,
x tick label style={rotate=45,anchor=east, font=\small},
 axis lines*=left, % Don't display the top and right lines
 symbolic x coords={antlr,fop,hsqldb,jython,chart,luindex,xalan,lusearch, pmd, eclipse}
]
\addplot coordinates 
	{(antlr,2.3781514264) (fop,1.905707750) (hsqldb,2.7388250106) (jython,1.8401924419) (chart,3.1126870659) (luindex,1.6126736676)
	(xalan,1.7175679043) (lusearch,2.2171653048) (pmd,1.3171653048) (eclipse,3.3822104461) };
\addplot coordinates 
	{(antlr,2.5781514264) (fop,1.945707750) (hsqldb, 2.7818250106) (jython,2.0401924419) (chart,3.6326870659) (luindex,1.912632376)
		(xalan,1.9375679043) (lusearch,2.4071653048) (pmd,1.3999716530) (eclipse,3.7922104461) };
\addplot coordinates 
	{(antlr,3.9781514264) (fop,3.605707750) (hsqldb,4.2388250106) (jython,3.3401924419) (chart,4.9126870659) (luindex,3.4126736676)
		(xalan,3.5175679043) (lusearch,4.1071653048) (pmd,3.1071653048) (eclipse,5.2922104461) };
\legend{Handwritten JVMTI, Our approach, Heap Dump + Eclipse MAT}
\end{axis}
\end{tikzpicture}
\caption{Analysis time with large input size\label{fig:analysisTimeLargeSize}}
 \end{minipage}
\hspace{1cm}
\end{figure*}

Figures~\ref{fig:analysisTimeDefaultSize} and~\ref{fig:analysisTimeLargeSize} present the results of the experiments.
In both cases, default and large input size, our approach is in between the handwritten JVMTI agent and the Eclipse MAT approach.
In comparison to Eclipse MAT, our approach reduces the analysis time by 25\% and 39\% for default and large input size respectively.
As expected, the analysis time increases with the number of objects, the slowdown shown between default and large input size is of 8.42\%.

\subsection{Profiling Time in Real Scenarios}
To evaluate the overhead of our approach in actual applications, 
we compute the memory consumption of bundles in real OSGi-based systems.
Since OSGi is a widely used framework, we chose applications built on top of OSGi or supporting it.
The custom profiler definition is based on the idea that bundle consumption is the consumption of a Java classloader.
Such a strategy is common when measuring memory consumption for Java-based component frameworks because modules are often isolated and represented through classloaders.
The complete profiler's definition is shown below:
\begin{lstlisting}[escapeinside={(*}{*)},
caption={Calculating the consumption of top components.},
label=topcomponents,
language=DSL2]
create structure foreach e:classloaders using
  constructor
    initialObjects = #[e]
    size = 0
  membership  ((ref_kind == root and this.class.classloader in this_structure) or
	(ref_kind != root and referrer in this_structure))
  updates
    size = size + this.size
\end{lstlisting}

This experiment aims at evaluating the profiling time for each application using our approach and \textit{Heap Dump + Eclipse MAT}.
In this experiment, each application is executed, once it is initialized, the memory profiler is invoked, and its execution time measured.
This process is repeated ten times for each application and analysis approach in order to use the average as final measurement.
We use \textit{Heap Dump + Eclipse MAT}  to compute the memory retained for top level classloaders using  a standard analysis named \textit{top components} reports. 

To execute the memory analysis from within the applications, we implemented extensions for each application (e.g., an Eclipse plugin, a NetBean module). ~\footnote{ The evaluation code is available online: \url{https://github.com/intigonzalez/heapexplorer\_language}}.

These extensions are in charge of triggering the analysis.
It was necessary because in our approach the analysis must be executed by the JVM that is being profiled.
In this experiment, we perform the analysis on the following systems: Eclipse Luna~\cite{luna}, NetBeans 8.0\cite{netbeans}, dotCMS 3.1~\cite{dotcms}, Cytoscape 3.2.1~\cite{cytoscape}, Glassfish 4.1~\cite{glassfish},  Liferay 6.2.2~\cite{liferay}, and WildFly 8.2~\cite{wildfly}.

\begin{figure}[!h]
\centering
\begin{tikzpicture}
\begin{axis}[ybar=0pt, legend style={at={(0.72,1)},
every axis legend/.append style={nodes={right}},
anchor=north,legend columns=1, font=\tiny},
ylabel={Analysis Time (sec)},
y label style={at={(0.02, 0.5)}},
scaled y ticks = false,
      y tick label style={/pgf/number format/fixed,
      /pgf/number format/1000 sep = \thinspace % Optional if you want to replace comma as the 1000 separator 
      },
xtick=data,ymin=0,
width = 0.8\columnwidth,
height = 4.2cm,
bar width = 7,
x tick label style={rotate=45,anchor=east, font=\small},
 axis lines*=left, % Don't display the top and right lines
 symbolic x coords={Eclipse Luna, NetBean 8.0, dotCMS 3.1,Cytoscape 3.2.1,Glassfish 4.1, Liferay 6.2.2, WildFly 8.2}
]
\addplot coordinates 
	{(Eclipse Luna,3.9781514264) (NetBean 8.0, 4.605707750) (dotCMS 3.1, 9.2388250106) (Cytoscape 3.2.1, 1.3401924419) (Glassfish 4.1, 2.9126870659) (Liferay 6.2.2,4.9126870659) (WildFly 8.2, 3.9126870659) };
\addplot coordinates 
	{(Eclipse Luna,42.133233423) (NetBean 8.0,38.388906289) (dotCMS 3.1,30.9167577408) (Cytoscape 3.2.1,25.99) (Glassfish 4.1, 18.46) (Liferay 6.2.2, 28.9126870659) (WildFly 8.2, 19.9126870659)};
\legend{Our Approach, Heap Dump + Eclipse MAT}
\end{axis}
\end{tikzpicture}
\caption{Analysis time for real applications. It shows the time needed to compute an analysis just once. The analysis aims at finding the consumption of the top components\label{fig:analysisTime}}
\end{figure}

Figure~\ref{fig:analysisTime} presents the analysis time for several applications and two analysis approaches.
Our approach outperforms Eclipse MAT in all applications; the gain is 3x-19x with an average of 8x.
Two factors influence the measurements.
First , Eclipse MAT invests some time parsing the dump file, and creating the internal indexes to accelerate queries' response time.
Second, the \textit{top components} report in Eclipse MAT can only be implemented, using its query language, in terms of the function \textit{retainedHeapSize}, which calculates the amount of memory retained for a given object.
Since this function is costly to compute, Eclipse MAT spends a considerable amount of time on it while building the \textit{top components} report.

%The results shown in figure~\ref{fig:evaluation} confirm the conclusions already discussed.
%Furthermore, they gave an initial estimation of the baseline overhead we can expect when the DSL approach is used.

\section{ Conclusions } \label{sec:conclusions}

In this chapter, we propose a Domain Specific Language for expressing the mapping between abstractions and runtime data structure to collect information about the memory heap in production.
This language provides an abstraction that is useful to reason about the heap and is, at the same time, easy to translate into a set of low-level routines to efficiently collect the desired information.
In our opinion, this approach is a step forward in the creation of resource-aware software systems for two reasons. 
First, it reduces the complexity of defining customized queries; hence, developers and operators are able to use this feature to solve new problems without the need of high expertise on runtime internals.
Second, such customized queries can be used in a production environment since they have a limited impact on the system's performance.

The approach proposed in this chapter contributes to answer two research questions presented in the introduction of this thesis (see Section \ref{sec:intro-challenges}).
In particular, it answers \textit{\ref{rq:rq1}} (\textit{How can we provide portable and efficient support for resource consumption monitoring?}) and \textit{\ref{rq:rq4}} (\textit{How can we ease the definition and implementation of monitoring tools for new software abstractions?}) by defining a metalanguage to describe the behavior of customized memory profilers.
These profilers are useful to efficiently calculate at runtime how components and other domain-specific abstractions consume resources. 
