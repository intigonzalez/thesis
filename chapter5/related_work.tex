\section{Related work}\label{sec:related}

The Scapegoat framework is related to component monitoring, Models@run.time, component isolation and component performance prediction approaches.

Performance and resource-consumption prediction approaches are complementary to the Scapegoat framework because they can assist in better specifying the component contracts.
Some approaches require developers to provide extensive per-component metadata at design-time in order to calculate the application's overall performance or resource consumption \cite{Becker:2007:MPP:1216993.1217006,Jonge03scenario-basedprediction}.
Prediction approaches have been achieved by using combinations of design-time and runtime analyses \cite{autili2012hybrid}.
However, although many approaches to performance prediction have been proposed, none of them have obtained widespread use \cite{Koziolek:2006:QDD:2171366.2171393}.

%In \cite{Ghezzi2009}, the authors propose using performance models based on Queuing Networks to cope with the problem of architectural reasoning.
%The authors keep Models@run.time to enable re-estimation of model parameters based on the behavior of the running system and they present a mechanism to update the parameters when usage profiles change.
%The authors target the same problem we are dealing with, although we are not facing the prediction of behavior.

KAMI \cite{Ghezzi2009} builds performance models at design-time but uses and continually refines them at runtime.
By collecting runtime data, they are able to build performance and resource consumption models that reflect real usage.
They are able to adapt the application according to changes in components' behavior, but they do not use nor propose an adaptive monitoring system to minimize overhead.

%\todo{For Walter: Review this paragraph}
State of the art monitoring systems \cite{FrenotS04,KregerHaroldWilliamson03,Binder200645} extract steady data-flows of system parameters, such as, the time spent executing a component, the amount of I/O and memory used, and the number of calls to a component.
The overhead that these monitoring systems introduce into applications is high, which makes it unlikely for them to be used in production systems.
Maurel et al. \cite{Maurel:2012:AME:2304736.2304763} propose an adaptive monitoring framework for the OSGi platform.
Similar to our approach, they propose a global monitoring system that changes to a localized monitoring system when a problem is detected.
However, their work is focused on CPU usage and does not consider other resources, such as memory or I/O.
Exploring the Java heap to obtain useful information about resource consumption has been proposed in~\cite{Price:2003:GCM:829515.830545, Geoffray5270296}.
As in our work, they account objects to the resource principal being explored (in their case to OSGi bundles) the first time an object is reached.
Their solutions modify the garbage collector in order to reduce overhead, but this causes resource accounting to be tied to, and performed on, each collection cycle.
%but it means that the accounting step is performed on each collection cycle.
In contrast, our approach can be executed on demand, albeit at the cost of further passes over the heap.
%although our accounting step requires further passes over the heap, we can precisely specify when to execute it.

Gama and Donsez \cite{Gama:2010:SCS:2176905.2176915} propose using virtual machines in separate processes or using MVM isolates \cite{czajkowski2012multitasking} to manage trusted and untrusted components.
After an evaluation period, untrusted components can be moved to the trusted JVM if no problems are detected.
This allows the main application to depend on potentially faulty components without risking severe crashes.
We can also cite Microsoft technologies such as COM (Component Object Model) components which can be either loaded in the client application process or provided in an isolated process \cite{lowy2001and}.
In addition to process virtualization, some operating systems also propose user-space virtualization, which isolates not only the processes but also the memory, the network interface and the file system. Examples of these approaches are Jails\footnote{\url{http://www.freebsd.org/doc/handbook/jails.html}} for BSD, LXC\footnote{\url{http://lxc.sourceforge.net/}} and CGroups for Linux, and lmctfy\footnote{\url{https://github.com/google/lmctfy}}.
All of these approaches have the drawback of limiting code and instance sharing and introduce additional overhead in cross-boundary component interactions.
Furthermore, depending on the complexity of the approach, there is also overhead in having to manage multiple processes.