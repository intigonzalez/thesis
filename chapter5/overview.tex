\section{Overview of the approach}


\subsection{Kevoree at a glance}
% Se proteger un peu plus sur Kevoree pour big node / placer plus de ref / comparaison ?
\noindent Kevoree is an open-source dynamic component model,\footnoteremember{kevoree}{http://www.kevoree.org\label{note:kevoree}} which relies on models at runtime~\cite{BlairBF09} to properly support the dynamic adaptation of distributed systems. Models@runtime basically pushes the idea of reflection~\cite{morin09a} one step further by considering the reflection layer as a real model that can be uncoupled from the running architecture (e.g. for reasoning, validation, and simulation purposes) and later automatically resynchronized with its running
instance. 
% Current Models@run.time approaches~\cite{models@runtime} provide higher-level abstractions and control on top of existing component-based platforms running on powerful nodes (Java-based), with no real support for distribution. Kevoree, as described in the following sub-sections, provides a solution to address the distributed and sporadic environment typically encountered in IoT infrastructures.


 Kevoree has been influenced by previous work that we carried out in the DiVA project~\cite{morin09a}.
With Kevoree we push our vision of Models@run.time~\cite{morin09f} farther.  
In particular, Kevoree provides a proper support for distributed Models@run.time. 
To this aim we introduce the \emph{Node} concept to model the infrastructure topology and the \emph{Group} concept to model semantics of inter node communication during synchronization of the reflection model among nodes.
Kevoree includes a \emph{Channel} concept to allow for multiple communication semantics between remote\emph{Components} deployed on heterogeneous nodes. 
All Kevoree concepts (Component, Channel, Node, Group) obey the object type design pattern~\cite{johnson_type_1997} to separate deployment artifacts from running artifacts.  
Kevoree supports multiple kinds of execution node technology (e.g.~Java, Android, MiniCloud, FreeBSD, Arduino, \dots\footnoterecall{kevoree}). 


\subsection{Dynamic Adaptation with Kevoree}
Kevoree aims at providing advanced adaptation capabilities to different types of nodes:
\begin{itemize}
\setlength{\itemsep}{0pt}
\setlength{\parskip}{0pt}
\setlength{\parsep}{0pt}
\item 
\noindent{\bf Level 1: Parametric adaptation.} Dynamic update of parameter values, e.g. change of sampling rate in a component that wraps a physical sensor (adaptation of instance properties).

\item 
\noindent{\bf Level 2: Architectural adaptation.} Dynamic addition or removal of bindings or components, e.g. replication of software components and channels on different nodes to perform load balancing (adaptation of instances graph).

\item 
\noindent{\bf Level 3: Dynamic provisioning of types.} Hot deployment of component types that were not foreseen before the initial deployment of the system. 
This allows for system evolution by enabling parametric and architectural reconfigurations, including management of instances for types that are added and managed dynamically (adaptation of types).

\item 
\noindent{\bf Level 4: Adaptation for remote management.} Nodes supporting level~4 adaptation participate in a remote management layer, which supervises less powerful nodes. 
This layer monitors remote nodes by requesting their current Kevoree model;
the layer triggers dynamic adaptation of nodes by sending precomputed reconfiguration scripts to them. 
This remote adaptation process supports seamless management of less powerful nodes by a more powerful one, which has enough resources to build and evaluate new and appropriate  configurations.
\end{itemize}

The adaptation engine relies on a model comparison between two Kevoree models to compute a  script for a safe system reconfiguration;
execution of this script brings the system from its current configuration to the new selected configuration~\cite{morin09a}. 
Model comparison yields  a delta-model defining changes (using CRUD operations) that should be applied on the source model to obtain the target model. 
Planification algorithms~\cite{daubert} use this delta-model as input in order to defined an efficient schedule of the adaptation steps. 
The delta-model is finally compiled into a Kevoree script. 
The Kevoree Script language (KevScript for short) is a core language for describing reconfiguration.
KevScript  is comparable to FScript for Fractal Component Model~\cite{DBLP:journals/adt/DavidLLC09}. 
Execution of a KevScript directly adapts a Kevoree system, without the need for  a full Kevoree model definition. 
Such adaptation scripts are written by designers, or they can be generated  by automated processes ({\em e.g.} within a  control loop managing the Kevoree system).
